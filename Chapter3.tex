\chapter{النتائج و المناقشة } % Main chapter title

\label{Chapter3} % Change X to a consecutive number; for referencing this chapter elsewhere, use \ref{ChapterX}

%----------------------------------------------------------------------------------------
%	SECTION 1
%----------------------------------------------------------------------------------------

\section{GaN نتريد الغاليوم }
\subsection{الخصائص الإلكترونية }

في هذا الجزء سنقوم بعرض ومناقشة عصابات الطاقة و كثافة الحالات الكلية لمركب نتريد الغاليوم GaN ، مع تحديد الطبيعة الالكترونية لهذه المادة ، و لدراسة الطبيعة الالكترونية لهذا المركب سنقوم بتحديد مساهمة كل من المدارات الالكترونية للذرات المشكلة لهذا المركب في المجال == على عصابات الطاقة في النقاط عالية التناظر عند منطقة بريلوين الأولى للمركب في تقريبي GGA و GGA+U المعرفة في الفضاء الطاقوي للشبكة المعكوسة و المميزة و المحصور بين ( 4.0- إلى eV 4.0 == ) 
باستخدام طريقة الأمواج المستوية المتزايدة خطيا والكمون الكامل FP-LAPW تحصلنابنقاط عالية التناظر موضحة في الشكل \ref{fig:bandgan}.

\subsubsection{عصابات الطاقة}

مجموع المنحنيات  $ E(K) $ تمثل منحنيات تشتت الالكترونات في البلورة و تدعى أيضا بنية عصابة الطاقة والتي تمثل كل حالات الطاقة المتاحة للالكترونات، فمن خلال ملأ الحالات بالكترونات البلورة عند الصفر المطلق نجد : حالات عصابة النقل ليست مملوءة (شاغرة) على عكس عصابة التكافؤ التي تكون مملوءة بالالكترونات لأن عصابات الطاقة تعطي الطاقات الممكنة للالكترونات كتابع لشعاع الموجة K و تمثل في الفضاء العكسي ولكن للتبسيط يتم التعامل فقط مع المتجهات ذات التناظر العالي في منطقة بريلوين الأولى وجدنا أن المركبين السابقين هما مركبين معدنبين بالنسبة لسبين أعلى ونصفا ناقل بالنسبة لسبين أسفل الشكل (  )، من الواضح أن النهاية الحدية الصغرى لعصابتي النقل $ ( CBM)  $مفصولة عن النهاية الكبرى لعصايتي التكافؤ $ ( VBM ) $ على طول المتجه (WK) بالنسبة لسبين أعلى ، كما لاحظنا أن بنية عصابة الطاقة للمركب. أما بالنسبة لسين أسفل وحدنا فحوة الطاقة $ Eg ( gap ) $ حول مستوى فارمي $ E_{F} $ أدى الى استقطاب سبيني ‫‪٪‬‬100 عند مستوى فارمي $ E_{F} $ ، ومنه المركبين يسلكان سلوك نصف معدن في الحالة الأساسية (equilibrium Me ) . قيمة الفجوة الممنوعة $ Eg ( gap ) $ للمركب GaN تساوي  ( == ) علي الترتيب ، تحصلنا على نتائج مماثلة أثناء حسابنا بتقريب GGA+U وهذا ما يثمن نتانجنا ، مع ذكر أنه حتى اليوم لم تثبت أي دراسة نظرية أو تجريبية للخصائص الالكترونية لهذه المركبات.يوضح الشكل \ref{fig:bandgan} بنية عصابات الطاقة لمركب  نتريد الغاليوم $ GaN  $ في تقريبي $ GGA $  حيث مستوى فارمی موجود عند $0.0 eV $.

\begin{figure}[h!]
	\centering
	\includegraphics[width=0.7\linewidth, height=0.3\textheight]{Fig/Fig_III/band_GaN}
	\caption{}
	\label{fig:bandgan}
\end{figure}
\FloatBarrier

\subsubsection{كثافة الحالات الكلية و الجزئية}

تم حساب كثافة الحالات الكلية ( DOS) والجزئية (TDOS) للمركب $ GaN  $  في تقريب +GGA بطريقـة ( Tetrahedron ) حيث استخدمنا عدد من النقاط الخاصة ( ==== نقطة لكل مركب).

بين الشكل \ref{fig:dosgan} كثافة الحالات الكلية والجزئية للمركب في تقريب GGA. يمثل مساهمة المستويات الذرية ( =5d 3d = ) و في حالتي جهتي مبين أعلى أو مبين اسفل في البناء الالكتروني للمركب ، حيث المستويان 3d و 5d ينقسما إلى مستويان ذريان هما و پيا و ذلك لكل ذرات المعلان الانتقالية: الكوبالت Co و المنغنز Mn و التانتال Ta و المستوي و لكل من ذرتي السيليسيوم Si أو الجرمنيوم Ge عموما المركبين لهما نفس الشكل العام لكثافة الحالات الكلية و الجزئية، بوجود عصاية تكافؤ وحيدة VB ،أي المجال الطاقي من - 4 إلى مستوى فارمي $ E_{F} $ ، يعود تشكل هذه العصابة إلى التحام المستويين الذربين Mn-te و Conte مع المستوي الذري Si-s للعنصر Si. حول مستوى فارمي يظهر جليا أن مساهمة المستويين الذريين  يكون كبير مقارنة مع المستويين Mne و Tat اللذان تكون مساهمتهما جزئية ، أما المستويين Coe و Tae فتكون مساهمتهما شبه منعدمة (سبين ) ، أما بالنسبة لسبين إتكون كثافة الحالات منعدمة حول مستوى فارمي $ E_{F} $ بسبب انتقال الالكترونات من المستوي الذري Mn t إلى المستوي الدربي Tae . وهذا ما يثبت أن المركب له صفة نصف معدن . من مستوى فارمی $ E_{F} $ إلى باقي المجال أي عصابة النقل CB نلاحظ مساهمة المستويين Min-tax و Ta-tg تكون كبيرة و باقي المستويات تكون مساهمتها شبه منعدمة (سبين)) أما بالنسبة لسبين لاحظنا مساهمة المستويات كبيرة و باقي المستويات تكون مساهمتها ضعيفة ، كما لا حظنا أن نسبة مساهمة المستويات الذرية للعنصر Ta تكون كبيرة في عصابة النقل CB.

\begin{figure}[h!]
	\centering
	\includegraphics[width=0.8\linewidth, height=0.4\textheight]{Fig/Fig_III/dos_GaN}
	\caption{}
	\label{fig:dosgan}
\end{figure}
\FloatBarrier

\subsection{الخصائص الضوئية}

أثبتت دراسة الخصائص الضوئية للمواد الصلبة أنها طريقة تمكننا من فهم الخصائص الإلكترونية للمواد. كفجوات الطاقة، حركة الإلكترونات بين شرائطها وذلك بتحليل الظواهر الضوئية المتمثلة في الامتصاص، الانعكاس، الانكسار والنفاذ.

يمكن معرفة الخواص الضوئية للمادة بمعرفة دالة السماحية :

يعطي الشكل ( == ) أطياف الجزء التخيلي (4) ع لدالة السماحية للمركب نتريد الغاليوم  $ GaN  $  في المجال الطاقوي eV[14 - 0] باستعمال تقريب $ GGAPBESol $ .
بالنسبة للمركب . 

\subsubsection{ثابت العزل الكهربائي }

\begin{figure}[h!]
	\centering
	\includegraphics[width=0.8\linewidth, height=0.4\textheight]{Fig/Fig_III/Dielectric_GaN}
	\caption{}
	\label{fig:dielectricgan}
\end{figure}
\FloatBarrier

\subsubsection{الناقلية و الامتصاص}

\begin{figure}[h!]
	\centering
	\includegraphics[width=0.8\linewidth, height=0.4\textheight]{Fig/Fig_III/Con_Abs_GaN}
	\caption{}الشكل ( == ) ، 
	\label{fig:conabsgan}
\end{figure}
\FloatBarrier

\subsubsection{معامل الإنكسار و معامل الخمود}

\begin{figure}[h!]
	\centering
	\includegraphics[width=0.8\linewidth, height=0.4\textheight]{Fig/Fig_III/Ref_Ext_GaN}
	\caption{}
	\label{fig:refextgan}
\end{figure}
\FloatBarrier

\section{    نتريدات الغاليوم المطعمة باليوروبيوم EuGaN }
\subsection{الخصائص الإلكترونية }

\subsubsection{عصابات الطاقة}

\begin{figure}[h!]
	\centering
	\includegraphics[width=0.8\linewidth, height=0.5\textheight]{Fig/Fig_III/EuGaNu_band}
	\caption{}
	\label{fig:euganuband}
\end{figure}
\FloatBarrier
\begin{figure}[h!]
	\centering
	\includegraphics[width=0.8\linewidth, height=0.5\textheight]{Fig/Fig_III/EuGaNd_band}
	\caption{}
	\label{fig:euganuband}
\end{figure}
\FloatBarrier
\subsubsection{كثافة الحالات الكلية و الجزئية}
\begin{figure}[h!]
	\centering
	\includegraphics[width=0.8\linewidth, height=0.5\textheight]{Fig/Fig_III/EuGaN_dos}
	\caption{}
	\label{fig:eugandos}
\end{figure}
\FloatBarrier
\subsection{الخصائص الضوئية}
\subsubsection{ثابت العزل الكهربائي }
\begin{figure}[h!]
	\centering
	\includegraphics[width=0.8\linewidth, height=0.5\textheight]{Fig/Fig_III/Dielectric_EuGaN}
	\caption{}
	\label{fig:dielectriceugan}
\end{figure}
\FloatBarrier
\subsubsection{الناقلية و الامتصاص}
\begin{figure}[h!]
	\centering
	\includegraphics[width=0.8\linewidth, height=0.5\textheight]{Fig/Fig_III/Con_Abs_EuGaN}
	\caption{}
	\label{fig:conabseugan}
\end{figure}
\FloatBarrier
\subsubsection{معامل الإنكسار و معامل الخمود}
\begin{figure}[h!]
	\centering
	\includegraphics[width=0.8\linewidth, height=0.5\textheight]{Fig/Fig_III/Ref_Ext_EuGaN}
	\caption{}
	\label{fig:refexteugan}
\end{figure}
\FloatBarrier
\chapter*{  خاتمة }