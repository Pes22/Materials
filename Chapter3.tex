\chapter{النتائج و المناقشة } % Main chapter title

\label{Chapter3} % Change X to a consecutive number; for referencing this chapter elsewhere, use \ref{ChapterX}

%----------------------------------------------------------------------------------------
%	SECTION 1
%----------------------------------------------------------------------------------------
مقدمة :
الهدف من هذا الفصل هو دراسة الخصائص الالكترونية كبنية عصابات الطاقة و كثافة الحالات الكلية والجزيئة ، و الخصائص الضوئية المتمثلة في الامتصاص، الانعكاس، الانكسار و معامل الخمود لمركب نتريد الغاليوم $ GaN $ المتبلور في بنية مكعب ممركز الاوجه $ (CFC)  $مع تناظر $ (F-43m ===) (زمرة فراغية 216===)  $كما هو موضح في الشكل  ثم قمنا في الشطر الثاني من الفصل التطبيقي بتطعيمه بعنصر من عناصر الارض الناذرة المتمثل في اليوروبيوم $ ( Eu ) $ حيث تحصلنا على مركب نتريدات الغاليوم المطعمة باليوروبيوم $ EuGaN $ . تم الحصول على النتائج في سياق نظرية دالية الكثافة $ DFT $ وذلك باستخدام طريقة الموجة المستوية المتزايدة خطيا والكمون الكامل $ (FP-LAPW) $ والمدمجة في برنامج $ Wien2k $ .

\begin{figure}[h!]
	\centering
	\includegraphics[width=0.5\linewidth, height=0.2\textheight]{Fig/Fig_III/Wurzite}
	\caption{}
	\label{fig:wurzite}
\end{figure}

من اجل دراسة الخصائص الإلكترونية و الضوئية استخدمنا التقريبين التاليين : تقريب التدرج المعمم $ (GGA) $ و تقريب $ (LDA) $ ، ومن اجل الحصول على نتائج جيدة نقوم بتحسين المانع الطاقي باستعمال التقريب المعدل لبيك جونسون (mBJ) في منطقة بريليون على ان عدد النقاط (==) .
		أول خطوة في الحساب هي إيجاد المعطيات التالية :

\begin{itemize}
	\item 
	تحديد التوزيع الإلكتروني لكل عنصر كيميائي داخل في تركيب المواد المدروسة ، حيث يبين الجدول  التوزيع الالكتروني لعناصر $ Eu $  ،$ Ga $ ، $ N $ .
	
	\begin{table}
		\centering
		\begin{tabular}{lll}
			\hline\noalign{\smallskip}
			RE &~~ Z~~ &~ El. Conf. \\
			\noalign{\smallskip}\hline\noalign{\smallskip}
			Eu &~~ 63~ & $ 4f^{7} 5d^{0} 6s^{2} 6p^{0} $ \\
			Ga &~~ 31~ & $ 3p^{6} 3d^{10} 4s^{2} 4p^{1}  $  \\
			N &~~ 07~ & $ 1s^{2} 2s^{2} 2p^{3} $\\
			\noalign{\smallskip}\hline
		\end{tabular}
		\caption{ يبين التوزيع الإلكتروني للعناصر المكونة للمركب المدروس .  }
		\label{tab:1}
	\end{table}

		\item 

		 اختيار قيمة $  R (نصف قطر الكرة MT) $ لذرات $ Eu $  ،$ Ga $ ، $ N $  و ذلك يعتمد على شرطين اساسیان :
		\begin{enumerate}
			\item 
			تجنب تداخل كرة $ (MT) $.
			\item 
			يجب أن تكون أغلبية الالكترونات القلبية داخل الكرة $ (MT) $.

		\end{enumerate}
		
			\item 
			معامل $ RK $ 
				\item 
				اصغر قطر للكرة $ RAT $
				\item
				 $ Komx $ الشعاع الناظمي للموجة المستوية
\end{itemize}


\section{ نتريد الغاليوم $ GaN $}
\subsection{الخصائص الإلكترونية }

في هذا الجزء سنقوم بعرض ومناقشة عصابات الطاقة و كثافة الحالات الكلية لمركب نتريد الغاليوم $ GaN $ ، مع تحديد الطبيعة الالكترونية لهذه المادة ، و لدراسة الطبيعة الالكترونية لهذا المركب سنقوم بتحديد مساهمة كل من المدارات الالكترونية للذرات المشكلة لهذا المركب في المجال == على عصابات الطاقة في النقاط عالية التناظر عند منطقة بريلوين الأولى للمركب في تقريبي $ GGA $ المعرفة في الفضاء الطاقوي للشبكة المعكوسة و المميزة و المحصور بين ( 4.0- إلى eV 4.0 == ) 
باستخدام طريقة الأمواج المستوية المتزايدة خطيا والكمون الكامل FP-LAPW تحصلنابنقاط عالية التناظر موضحة في الشكل \ref{fig:bandgan}.

\subsection*{عصابات الطاقة}

مجموع المنحنيات  $ E(K) $ تمثل منحنيات تشتت الالكترونات في البلورة التي تمثل كل حالات الطاقة المتاحة للالكترونات، فمن خلال ملأ الحالات بالكترونات البلورة عند الصفر المطلق نجد : حالات عصابة النقل ليست مملوءة (شاغرة) على عكس عصابة التكافؤ التي تكون مملوءة بالالكترونات لأن عصابات الطاقة تعطي الطاقات الممكنة للالكترونات كتابع لشعاع الموجة K و تمثل في الفضاء العكسي ولكن للتبسيط يتم التعامل فقط مع المتجهات ذات التناظر العالي في منطقة بريلوين الأولى وجدنا أن المركب هو مركب (معدنبين ==) بالنسبة للف المغزلي الأعلى $ ( Spin Up ) $ ونصف ناقل بالنسبة للف المغزلي الاسفل $ ( Spin down ) $ الشكل (  )، من الواضح أن النهاية الحدية الصغرى لعصابتي النقل $ ( CBM)  $مفصولة عن النهاية الكبرى لعصايتي التكافؤ $ ( VBM ) $ على طول المتجه $ (WK) $ بالنسبة للف المغزلي الأعلى $ ( Spin Up ) $ . أما بالنسبة لسين أسفل وحدنا فحوة الطاقة $ Eg ( gap ) $ حول مستوى فارمي $ E_{F} $ أدى الى استقطاب سبيني ‫‪٪‬‬100 عند مستوى فارمي $ E_{F} $ ، ومنه المركب يسلك سلوك نصف معدن في الحالة الأساسية $ (equilibrium Me ) $ . قيمة الفجوة الممنوعة $ Eg ( gap ) $ لمركب $ GaN $ تساوي  ( == ) ، يوضح الشكل \ref{fig:bandgan} بنية عصابات الطاقة لمركب  نتريد الغاليوم $ GaN  $ في تقريبي $ GGA $  حيث مستوى فارمی موجود عند $0.0 eV $.

\begin{figure}[h!]
	\centering
	\includegraphics[width=0.7\linewidth, height=0.3\textheight]{Fig/Fig_III/band_GaN}
	\caption{ عصابة الطاقة لمركب نتريد الغاليوم $ GaN  $ . }
	\label{fig:bandgan}
\end{figure}
\FloatBarrier

\subsection*{كثافة الحالات الكلية و الجزئية}

تم حساب كثافة الحالات الكلية ( DOS) والجزئية (TDOS) للمركب $ GaN $  في تقريب +GGA بطريقـة ( Tetrahedron ) حيث استخدمنا عدد من النقاط الخاصة ( ==== نقطة لكل مركب). حيث يبين الشكل \ref{fig:dosgan} كثافة الحالات الكلية والجزئية للمركب في تقريب GGA ، أين يمثل مساهمة المستويات الذرية ( =5d 3d = ) و في حالتي جهتي اللف المغزلي أعلى ( Spin Up ) أو اللف المغزلي اسفل ( Spin down ) في البناء الالكتروني للمركب ، بوجود عصاية تكافؤ وحيدة VB ،أي المجال الطاقي من - 4 إلى مستوى فارمي $ E_{F} $ . 

\begin{figure}[h!]
	\centering
	\includegraphics[width=0.8\linewidth, height=0.4\textheight]{Fig/Fig_III/dos_GaN}
	\caption{ كثافة الحالات الكلية $ ( DOS) $ والجزئية $ (TDOS) $ لمركب نتريد الغاليوم  $ GaN $   }
	\label{fig:dosgan}
\end{figure}
\FloatBarrier

\subsection{الخصائص الضوئية}

أثبتت دراسة الخصائص الضوئية للمواد الصلبة أنها طريقة تمكننا من فهم الخصائص الإلكترونية للمواد. كفجوات الطاقة، حركة الإلكترونات بين شرائطها وذلك بتحليل الظواهر الضوئية المتمثلة في الامتصاص، الانعكاس، الانكسار والنفاذ.

يعطي الشكل ( == ) أطياف الجزء التخيلي (4) ع لدالة السماحية للمركب نتريد الغاليوم  $ GaN  $  في المجال الطاقوي eV[14 - 0] باستعمال تقريب $ GGAPBESol $ .
بالنسبة للمركب . 

\subsection*{ثابت العزل الكهربائي }

\begin{figure}[h!]
	\centering
	\includegraphics[width=0.8\linewidth, height=0.4\textheight]{Fig/Fig_III/Dielectric_GaN}
	\caption{ ثابت العزل الكهربائي لمركب نتريد الغاليوم  $ GaN $  }
	\label{fig:dielectricgan}
\end{figure}
\FloatBarrier

\subsection*{الناقلية و الامتصاص}

\begin{figure}[h!]
	\centering
	\includegraphics[width=0.8\linewidth, height=0.4\textheight]{Fig/Fig_III/Con_Abs_GaN}
	\caption{معامل الامتصاص لمركب نتريد الغاليوم  $ GaN $ }، 
	\label{fig:conabsgan}
\end{figure}
\FloatBarrier

\subsection*{معامل الإنكسار و معامل الخمود}
يمثل الشكل منحنى معامل الانكسار لبلورة « $ GaN $ » بدلالة طاقة الفوتون

من الشكل (===) نلاحظ أن قرينة الانكسار لنتريد الغاليوم تساوي بالتقريب ( 2,31 )،و أكبر قيمة لها تساوي بالتقريب(3.46) وهذا عند القيمة (eV) 3.44   أي  nm 360.4 ، أيضا أصغر قيمة لقرينة الانكسار تساوي بالتقريب (0.11) عند القيمة (eV)11.38أي nm 108.9.

=====================
منحنى معامل الانكسار

=======================

الانعكاسية (R)لبلورة « $ GaN $ »
=============
منحنى الانعكاسية

===============

نلاحظ أن أكثر قيمة للانعكاب تـاوي بالتقريب 0.67) عند القيمة 9.2 الكترون فولط أي 134.7Bum البلورة المدروسة ($ GaN $) لها فاصل طاقي غير مباشر، وهذه النتائج تتطابق مع دراسات نظرية سابقة .

 معامل الخمود:

تم حساب معامل الخمود لمركب نتريد الغاليوم  $ GaN $ وفقا للعلاقة (==1)، والشكل (==6) يبين تغير معامل الخمود كدالة لطاقة الفوتون ، نلاحظ أن منحنى معامل الخمود للمركب يقل بنسبة قليلة عند طاقات الفوتونية الضعيفة ثم يزداد بشكل سريع ومفاجئ في مدى الطاقات الفوتونية العالية. وهذه الزيادة ناتجة عن الزيادة السريعة لمعامل الامتصاص عند هذه الطاقات والتي تدل على حدوث انتقالات الكترونية مباشرة .
الفوتون ويكون أعظم ما يمكن عند الطاقات المقابلة لحافة الامتصاص الأساسية.

\begin{figure}[h!]
	\centering
	\includegraphics[width=0.8\linewidth, height=0.4\textheight]{Fig/Fig_III/Ref_Ext_GaN}
	\caption{ معامل الخمود لمركب نتريد الغاليوم  $ GaN $ }
	\label{fig:refextgan}
\end{figure}
\FloatBarrier

\section{  نتريدات الغاليوم المطعمة باليوروبيوم EuGaN }
\subsection*{الخصائص الإلكترونية }

\subsection*{عصابات الطاقة}
تحصلنا على نتائج مماثلة أثناء حسابنا بتقريب GGA+U وهذا ما يثمن نتانجنا ، مع ذكر أنه حتى اليوم لم تثبت أي دراسة نظرية أو تجريبية للخصائص الالكترونية لهذه المركبات.

\begin{figure}[h!]
	\centering
	\includegraphics[width=0.8\linewidth, height=0.5\textheight]{Fig/Fig_III/EuGaNu_band}
	\caption{ عصابات الطاقة لمركب $  EuGaN $ }
	\label{fig:euganuband}
\end{figure}
\FloatBarrier

\begin{figure}[h!]
	\centering
	\includegraphics[width=0.8\linewidth, height=0.5\textheight]{Fig/Fig_III/EuGaNd_band}
	\caption{ عصابات الطاقة لمركب $  EuGaN $ }
	\label{fig:euganuband}
\end{figure}
\FloatBarrier

\subsection*{كثافة الحالات الكلية و الجزئية}

بعد تطعيم مركب نتريد الغاليوم $ GaN $ بذرة الاروبيوم $ Eu $ تتغير كثافة الحالة الكلية والجزئية في المركب كما هو مبين في المخططات التالية :

\begin{figure}[h!]
	\centering
	\includegraphics[width=0.8\linewidth, height=0.5\textheight]{Fig/Fig_III/EuGaN_dos}
	\caption{ كثافة الحالة الكلية لمركب $  EuGaN $}
	\label{fig:eugandos}
\end{figure}
\FloatBarrier

من خلال ملاحظة منحنيات الكثافة الالكترونية الكلية أو الجزئية للمركبين قبل و بعد التطعيم أو لكل ذرة تدخل في تركيب هذين الجزيئين  فاننا نستنتج تنشيط مركب مركب نتريد الغاليوم $ GaN $ بذرة الاروبيوم $ Eu $ فيزيد ذلك في الكثافة .
حيث تم حساب كثافة الحالات الكلية $ ( DOS) $ والجزئية $ (TDOS) $ للمركب الناتج بعد التطعيم $ EuGaN $  في تقريب $ +GGA  $ بطريقـة $ ( Tetrahedron ) $ حيث استخدمنا عدد من النقاط الخاصة ( ==== نقطة لكل مركب). فبين لنا الشكل \ref{fig:dosgan} مساهمة المستويات الذرية ( =5d 3d = ) و في حالتي جهتي اللف المغزلي أعلى $ ( Spin Up ) $ أو اللف المغزلي اسفل $ ( Spin down ) $ في البناء الالكتروني للمركب ، بوجود عصاية تكافؤ وحيدة $ VB $ ،أي المجال الطاقي من - 4 إلى مستوى فارمي $ E_{F} $ . يعود تشكل هذه العصابة إلى التحام المستويين الذربين $ Mn-te $ و $ Conte $ مع المستوي الذري $ GaN-s  $ للعنصر $ Si $ . حول مستوى فارمي يظهر جليا أن مساهمة المستويين الذريين  يكون كبير مقارنة مع المستويين $ Mne $ و $ Tat $ اللذان تكون مساهمتهما جزئية ، أما المستويين $ Coe $ و $ Tae $ فتكون مساهمتهما شبه منعدمة (سبين ) ، أما بالنسبة لسبين إتكون كثافة الحالات منعدمة حول مستوى فارمي $ E_{F} $ بسبب انتقال الالكترونات من المستوي الذري Mn t إلى المستوي الدربي Tae . وهذا ما يثبت أن المركب له صفة نصف معدن . من مستوى فارمی $ E_{F} $ إلى باقي المجال أي عصابة النقل CB نلاحظ مساهمة المستويين Min-tax و Ta-tg تكون كبيرة و باقي المستويات تكون مساهمتها شبه منعدمة (سبين)) أما بالنسبة لسبين لاحظنا مساهمة المستويات كبيرة و باقي المستويات تكون مساهمتها ضعيفة ، كما لا حظنا أن نسبة مساهمة المستويات الذرية للعنصر Ta تكون كبيرة في عصابة النقل CB.

\subsection{الخصائص الضوئية}

\subsection*{ثابت العزل الكهربائي }

\begin{figure}[h!]
	\centering
	\includegraphics[width=0.8\linewidth, height=0.5\textheight]{Fig/Fig_III/Dielectric_EuGaN}
	\caption{ثابت العزل الكهربائي لمركب $  EuGaN $}
	\label{fig:dielectriceugan}
\end{figure}
\FloatBarrier

\subsection*{الناقلية و الامتصاص}

\begin{figure}[h!]
	\centering
	\includegraphics[width=0.8\linewidth, height=0.5\textheight]{Fig/Fig_III/Con_Abs_EuGaN}
	\caption{ معامل الامتصاص لمركب $  EuGaN $ }
	\label{fig:conabseugan}
\end{figure}
\FloatBarrier

\subsection*{معامل الإنكسار و معامل الخمود}
\begin{figure}[h!]
	\centering
	\includegraphics[width=0.8\linewidth, height=0.5\textheight]{Fig/Fig_III/Ref_Ext_EuGaN}
	\caption{ معامل الإنكسار لمركب $  EuGaN $  }
	\label{fig:refexteugan}
\end{figure}
\FloatBarrier

==============
منحنى معامل الخمود

==============

\subsection{الخصائص المغناطيسية}
في عملنا هذا تم دراسة الحالة المغناطيسية (FM) حيث تم حساب العزم المغناطيسي الكلي (Total) والجزئي في اللف المغزلي الاعلى ( Spin Up ) في حالة دالة الموجة النسبية باستخدام التقريب GGA المبينة في الجدول التالي :

==================
الجدول

==================


\chapter*{  خاتمة }