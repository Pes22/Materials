
\chapter{  نظرية دالة الكثــافة $ DFT $ و برنامج $ Wien2k $  } % 

\label{Chapter2} 
%----------------------------------------------------------------------------------------

% Define some commands to keep the formatting separated from the content 
%----------------------------------------------------------------------------------------

\section*{ مقدمـة }

تحتوي المواد الصلبة على الأنوية موجبة الشحنة و الإلكترونات سالبة الشحنة تتفاعل كهربائيا فيما بينها ( تفاعل نواة –نواة ,إلكرتون – نواة , إلكرتون – إلكرتون ). 
من أجل إيجاد الخصائص الفيزيائية و الكيميائية للأنظمة البلورية ومن أجل معرفة البنية الإلكترونية لهذه الأنظمة  نعتمد على حل معادلة  شرودينغر المتعددة الالكترونات إلا أن حل مثل هذه المسائل يكون معقد, لذلك نقوم بتبسيطها إلى نظام معادلات أحادية الإلكترونات ليسهل حلها عدديا . ونستعمل لهذا التقريب نظرية الكثافة التابعية وذلك باستخدام طريقة الامواج المستوية المتزايدة خطيا والكمون الكامل $ (FP-LAPW) $ والمدمجة في برنامج $ Wien2k $ .

\section{ ‫نظرية دالة الكثــافة‬ ‫‪$ DFT‬‬ $ $ ( density functionnal theory ) $ }

تعتبر نظرية دالة الكثافة أحد أهم الطرق المستعملة  في الفيزياء و الكيمياء النظريين  و بواسطتها نستطيع أن نحدد خصائص نظام متعدد الجسيمات ( الطاقة الكلية للنظام, الكثافة الإلكترونية للمدارات, المعاملات الفيزيائية والضوئية للمادة ...)   وهي واحدة من أكثر الطرق استخداما في العمليات الحسابية الكمومية  لحل معادلة شرودنجر بسبب إمكانية تطبيقها على أنظمة متنوعة و السرعة العالية, تستخدم ضمن برنامج Wien2k الذي يعمل تحت نظام التشغيل $ LUNIX  $ بلغة البرمجة " $ Fortan 90 $ " .
\\


\subsection{ معادلة شرودنغر $ ( Shrodinger's equation ) $ }

تعتبر معادلة  شرود نجر هي منطلق كل الدراسات الكمية للنظام الكوانتي للبلورات, حيث  يوصف نظام الجسيمات المتفاعلة من أيونات و إلكترونات بالمعادلة التالية :         	
\begin{equation}\label{}
 H \psi = E \psi 
\end{equation}
حيث :
\begin{list}{}{}
	\item 
	 . $ H  $ يمثل الهملتونيان 
	\item 
	  طاقة النظام $ E $ .
	\item 
	. $ \psi $ دالة الموجة 
\end{list}

الهملتونيان الكلي للجملة يكون مؤلف من الطاقة الحركية لكل الجسيمات وطاقة التفاعل فيما بينها و في بعض الحالات طاقة التفاعل مع الوسط الخارجي في غياب لحقل الخارجي يكتب الهملتون بالشكل التالي :
\begin{itemize}
	\item 
	\begin{equation}\label{}
		H = T_e + T_N + V_{ee} + V_{e-e} + V_{N-N}
	\end{equation}
	
	\item  
	الطاقة الحركية للإلكترونات :
	\begin{equation}\label{}
		T_{e} = \sum_{i=1}^n \frac{ P_{i} ^2 }{ 2m_{i} }¨ =  \sum_{i=1}^n -\frac{ \bar{h} ^{2}  }{ 2m_{i} }¨( \nabla_{i}^{2} )
	\end{equation}
	
	\item
	الطاقة الحركية للأنوية :
	\begin{equation}\label{}
		T_{N} = \sum_{i=1}^n \frac{ P_{\alpha} ^2 }{ 2m_{\alpha} }¨ =  \sum_{i=1}^n -\frac{ \bar{h} ^{2}  }{ 2m_{\alpha} }¨( \nabla_{\alpha}^{2} )
	\end{equation}
	
	\item
	طاقة التفاعل إلكرتون – إلكرتون :
	\begin{equation}\label{}
		V_{ee} = \dfrac{1}{4\pi \epsilon_{0} } \sum_{i‫≠‬j}^n \frac{1}{2} \dfrac{e^{2}}{(\vec{r}_{i} - \vec{r}_{j})}
	\end{equation}
	
	\item
	طاقة التفاعل إلكرتون – نواة :
	\begin{equation}\label{}
		V_{eN} = \dfrac{-1}{4\pi \epsilon_{0} } \sum_{i}^\alpha  \dfrac{e^{2}Z_{\alpha} e}{(\vec{r}_{i} - \vec{R}_{\alpha})}
	\end{equation}
	
	\item
	طاقة التفاعل نواة – نواة :
	\begin{equation}\label{}
		V_{NN} = \dfrac{1}{4\pi \epsilon_{0} } \dfrac{1}{2} \sum_{\alpha}^\beta  \dfrac{e^{2}Z_{\alpha} Z_{\beta} }{(\vec{R}_{\alpha} - \vec{R}_{\beta})}
	\end{equation}
	
\end{itemize}

حيث المعاملات :
\begin{list}{}{}
	
	\item 
	i و j : خاصة بالإلكترونات .
	\item 
	$ \beta $ و  $ \alpha $ :  .خاصة بالأنوية
	\item 
	 $ M $ و $ m $ : كتلة الأنوية  و الإلكترونات على الترتيب .
	\item 
	 $ e \alpha Z $  : شحنة الأنوية .
	\item 
	المسافة بين النواتين $ \beta $ و  $ \alpha $  : $(\vec{R}_{\alpha} - \vec{R}_{\beta})$	.
	\item 
	المسافة بين الإلكترون $ i $ و النواة $ \alpha $: $(\vec{r}_{i} - \vec{R}_{\alpha})$ 	. 
	\item 
	$(\vec{r}_{i} - \vec{r}_{j})$	المسافة  بين الإلكرتونين : $ i $ و $ j $ . 
\end{list}

في حالة السكون معادلة  شرودنجر تكون مستقلة عن الزمن بحيث :
\begin{equation}\label{}
	H \psi (r;R) = E \psi (r;R)
\end{equation}

بحيث $\psi$ تمثل دالة الموجة وتتعلق بكل إحداثيات الجسيمات سواء إلكترونات أو أنويه , E  القيم الذاتية الموافقة لمعادلة شرودنجر لـ  $ N $ذرة تحتوي على $  N ( Z + 1 )3 $ متغير , وبهذا تعتبر معادلة  شرودنجر مستحيلة الحل ومن أجل تبسيطها وتسهيل حلها وضعت عدة تقريبات.


\subsection{ تقريب بورن ابنهايمر $ ( Born Bnheymer's approximation ) $ }

نظرا لحركة الإلكترونات السريعة جدا بالنسبة لحركة الأنوية  يفترض هذا التقريب أن  النواة ساكنة بالنسبة للإلكترونات أي الطاقة الحركية لها معدومة  $T_{N} = 0$ وأن طاقة التفاعل بين الأنوية ثابتة $ V_{NN} = Cst $ ومنه يصبح الهلملتون H بالصيغة التالية :

\begin{equation}\label{}
	H_{e} = T_{e} + V_{ee} + V_{eN}
\end{equation}

وتصبح معادلة شرودنجر بالشكل التالي :
\begin{equation}\label{}
	\left[\frac{ \bar{h} ^{2}  }{ 2m_{\alpha} }¨\sum_{i=1}^n  \nabla_{i}^{2} + \dfrac{1}{2} \sum_{i , j}^n \dfrac{e^{2}}{r_{i} - r_{j}} - \sum_{i , \alpha}^n \dfrac {e^{2} Z \alpha }{r_{i} - R_{\alpha}}\right] \psi_{e} = E_{e} \psi_{e}
\end{equation}
لحل معادلة  شرودنجر نستعين بمجموعة من الطرق لهرتريفوك  الأساس فيها أن يكون الإلكترون  حر وتستخدم هذه الطريقة بكثرة في الكيمياء الكمية لدراسة الجزيئات و الذرات, بينما في الجسم الصلب فتستخدم عدة طرق حديثة وأكثر دقة وفعالية مثل نظرية الكثافة التابعية  $ DFT $ .

\subsection{   نظرية هوهنبارغ-كوهن $ ( Hohumberg-Kohan's Theory ) $ }

نظرية هوهانبورغ وكوهين هي نظرية الكثافة التابعية ترتكز على نظريتين هما هوهانبورغ  و كوهين .
\begin{itemize}
	\item 
	النظرية الأولى : يتم فيها تعريف الطاقة الكلية E لنظام من N إلكترون المتفاعلة  في الحالة الأساسية بوجود كمون خارجي للأنوية $  V_{ext} $ ,على أّنها دالة وحيدة للكثافة الإلكترونية Q(r) بالشكل التالي :
	
	\begin{equation}\label{}
		E(\rho) = F(\rho) + \int  \rho (r) V_{ext} (r) dr^{3} 	
	\end{equation}
	حيث :
	\begin{equation}\label{}
		F \left[(\rho)\right] = T \left[(\rho)\right] + V_{e-e}   \left[\rho(r)\right]
	\end{equation}
	مع :
	\begin{list}{}{}
		\item 
		$ F \left[(\rho)\right] $ : دالة شاملة للكثافة الإلكترونية .
		\item 
		T طاقة حركية .
		\item 
		$𝑉_{ 𝑒~~𝑒} $ :  طاقة التفاعل إلكرتون – إلكرتون.
	\end{list}
	\item
	النظرية الثانية : تظهر أن الكثافة الإلكترونية للحالة الأساسية توافق أقل قيمة للطاقة , وكل الخصائص الأخرى تكون تابعة لهذه  الكثافة . 
	\begin{equation}\label{}
E(\rho_{0}) = min E(\rho)
	\end{equation}
	مع  $ \rho_{0} $ هي كثافة الحالة الأساسية .  
\end{itemize}

\subsection{ معادلة كوهان-شوم  : $ (  kohnSham's Equation ) $ }

يتم فيها كتابة معادلة الطاقة بالشكل التالي :
\begin{equation}\label{}
	E \left[(\rho)\right] = T_{0} \left[\rho(\vec{r})\right] + E_{H} \left[\rho(\vec{r})\right] + E_{xc} \left[\rho(\vec{r})\right] + V_{ext} \rho(\vec{r}) d^{3}r
\end{equation}
حيث :
\begin{list}{}{}
	\item 
	$ T_{0} \left[\rho(\vec{r})\right] $  الطاقة الحركية للغاز الإلكتروني في الحالة الأساسية (الاستقرار) .
	\item 
	$ E_{H} \left[\rho(\vec{r})\right] $ حد هرتري للإلكترونات .
	\item 
	$ E_{xc} \left[\rho(\vec{r})\right] $  طاقة التبادل والإرتباط .
	\item 
	$ V_{ext} \rho(\vec{r})  $ كمون خارجي يؤثر على النظام الإلكتروني .
	
\end{list}

حل معادلة كوهن شوم : ترتكز معظم حسابات عصابات الطاقة على $ DFT $ وتترتب حسب استخدامها  للكثافة ,الكمون ومدارات كوهنشوم , ومن بينها طريقة الموجة المستوية المتزايدة خطيا $ LAPW-FP $ والتي تعتمد على  مدارات كوهن-شوم وتعطي معادلة الموجة الأساسية بالشكل التالي :
\begin{equation}\label{key}
	\psi_{i} (\vec{r}) = \sum C_{i\alpha} \phi_{\vec{r}} (r)
\end{equation}

حيث  $ C_{i\alpha} $ معامل النشر للدالة الموجية و $ \phi_{\alpha} (r) $ المعادلة الأساسية .
حل معادلة كوهن شوم يتطلب تعريف المعامل $ C_{i\alpha} $  لكل مدار مشغول بحيث تكون الطاقة الكلية في القيمة الدنيا ,وتطبق على النقاط عالية التناظر في منطقة  بريلوان الأولى لتسهيل الحساب و بحكم وجود تنافر بين الإلكترونات تستخدم حلقة  تكرارية ولأجل تحقيق التقريب المطلوب أدخلت الكثافة الأولية للشحنة  $ _{In}\phi  $ في الحساب .  
حلول معادلة كوهان-شوم تعطى : $ 0 = C_{i} ( H - \epsilon_{i} S ) $
حيث :
\begin{list}{}{}
	\item 
	H هملتونيان كوهان – شوم .
	\item
	S مصفوفة التغطية.
\end{list}

\subsection{ دالة تبادل-ارتباط $ ( fonctionnelle d'échange et corrélation ) $  }

طاقة التبادل والارتباط $ ( fonctionnelle d'échange et corrélation ) $ و التي يعبر عنها بالمعادلة التالية :
\begin{equation}\label{}
	E_{xc} \left[\rho\right] = F_{HK} \left[\rho\right] - T_{0} \left[\rho\right] - V_{H} \left[\rho\right]	
\end{equation}
ليس لها قيمة مضبوطة و لأن حل معادلة كون شام مرتبطة بطاقة التبادل و الارتباط فتعتمد عدة  تقريبات لإعطائها في شكل تحليلي و التقريب المستعمل بكثرة  في هذه الحالة تقريب الكثافة الكلية $ ( locale densité la de approximation'L ) LDA  $ و تقريب التدرج  المعمم $ ( L'approximation du gradient généralisé ) GGA $ .

\subsection{ تقريب كثافة الموضع ( $ LDA $ ) }
تقريب الكثافة الكلية $ LDA  $ تجرى الدراسة على نظام متجانس أو شبه متجانس ( غاز إلكتروني منتظم  حيث $ \rho $  ثابتة) طاقة التبادل والارتباط متعلقة فقط بالكثافة الإلكترونية في نقطة $ r $ بإهمال كل التأثيرات التي تجعل النظام غير متجانس ) نعتبر الكثافة ثابتة أو تتغير ببطء شديد( يعبر عن طاقة التبادل والار تباط لجزء $ \epsilon_{xc} $ بالشكل :
\begin{equation}\label{}
	E_{xc}^{LDA} \left[\rho\right] =  \int  \epsilon_{xc} \left[\rho (r) \right] \rho (r) d^{3}r
\end{equation}

تقريب $ LDA $   يعتبر $ \epsilon_{xc} $ طاقة كلية وهي مقسمة لجزئين بحيث : 
\begin{equation}\label{key}
	\epsilon_{xc} (\rho) = \epsilon_{c} (\rho) + \epsilon_{x} (\rho)
\end{equation}
مع $  \epsilon_{c} $ طاقة الارتباط و $ \epsilon_{x} $ طاقة التبادل .\\
هذا التقريب لا يستعمل إلا في حالة غاز إلكتروني منتظم و نعلم أنه في الأنظمة الحقيقية الكثافة الإلكترونية لا تكون منتظمة محليا في منطقة معينة ( في منطقة معينة ) لهذا السبب يستعمل في الغالب تقريب التدرج المعمم .

\subsection{ تقريب التدرج المعمم ( $ GGA $ ) }

أدخلت هذه التقنية لتحسين دقة النتيجة المتحصل عليها بحيث يتم كتابة طاقة التبادل و الارتباط كدالة للكثافة الإلكترونية $ \rho(r) $ والتدرج ( $ \nabla \rho(r) $ )  لا يؤخذ كخاصية منتظمة للغاز الإلكتروني .
\begin{equation}\label{}
	E_{xc}^{GGA} (\rho) =  \int f \left[\rho (r) \nabla \rho(r) \right] d^{3}r
\end{equation}

\begin{figure}[bh]
	\centering
	\includegraphics[width=0.65\linewidth, height=0.7\textheight]{Fig/Fig_II/DFT}
	\caption{ مخطط عام لنظرية كثافة الدالة $ DFT $ }
	\label{fig:dft-11}
\end{figure}


\section{ طريقة الامواج المستوية المتزايدة خطيا ( $ LAPW-FP $ ) }

\subsection{ طريقة الأمواج المستوية المتزايدة  ( $ La méthode linéaire des ondes planes augmentées : LAPW $ )}
تم عرضها من طرف أندرسون $ ( Andersen ) $ هي والكمون الكامل  $ LAPW-FP $ من أجل تحسين طريقة الموجة المستوية المتزايدة (  $ APW $ ) لسلتر $ ( SLATER )  $ ; و لكتابة دالة الموجة للإلكترونات أخذ سلتر شكل دالة الإلكترونات الخاصة بكمون ( خلية النحل ) أو ما يسمى بكمون $ ( T.M ) $ والذي يقسم الفضاء المحيط بالذرات إلى منطقتين :

\begin{figure}[h!]
	\centering
	\includegraphics[width=0.6\linewidth]{Fig/Fig_II/RMT}
	\caption{دائرتين توضحان كمون الكرة $ (T.M) $ }
	\label{fig:lapw1}
\end{figure}
\FloatBarrier
المنطقة الأولى : داخل كرة $ (T.M) $ تشمل كل من الأنوية و الإلكترونات القلبية شديدة الارتباط بها.

المنطقة الثانية : المنطقة الإقحامية تحيط بالكرات وتشمل الإلكترونات للمدارات الخارجية ضعيفة الإرتباط  بالأنوية .\\
حيث :
$ R_{\alpha} $ : 	يمثل نصف قطر الكرة $ (M.T) $ والتي تعطى بالعلاقة التالية :		
\begin{equation}\label{}
	\phi (r) = \left\{\begin{array}{rcl}
		\frac{1}{\sqrt{\Omega}} \sum C_{G} e^{i(G+K)r}r & \geqslant & R_{\alpha} \\
		\sum A_{lm} U_{l} (r) Y_{lm}(r) r & \geqslant & R_{\alpha}
	\end{array} \right.
\end{equation}

حيث $ \Omega $ : حجم خلية الوحدة ، $ Y_{lm} $ الدالة التوافقية للكروية، $ C_{G} $ معاملات النشر.\\
وتكون حلول معادلة الشرودنجر كالاتي :
\begin{enumerate}
	\item  
	حلول شعاعية داخل الكرة $ (M.T)  $ . 
	\item 
	موجة مستوية في المنطقة الإقحامية .
\end{enumerate}
و  $ U_{l} (r) $ هي حلول منتظمة لمعادلة شرودنجر للجزء الشعاعي الذي يكتب  :		
\begin{equation}\label{}
	\left\{\frac{d^{2}}{dr^{2}} + \dfrac{l(l+1)}{r^{2}} + V(r) - E_{l} \right\} r U_{l}^{(1)} (r) = 0 
\end{equation}
حيث : V(r) الكمون الكروي و $  E_{l} $  الطاقة الخطية .
ولضمان استمرار الدالة $ \phi (r) $ على سطح الكرة $ (T.M) $ تنشر المعاملات $ A_{lm} $ بدلالة المعاملات $ C_{G} $ الخاصة بالامواج المستوية في  المنطقة الاقحامية ، بعد الحساب الجبرية نجد :
\begin{equation}\label{}
	A_{lm} = \dfrac{4\pi i^{l}}{\sqrt{\Omega} U_{l} (R_{l})} \sum C_{G} j_{L} ( K + g  R_{\alpha}) Y*_{lm} (K+G)
\end{equation}


\subsection{ مبدأ طريقة الأمواج المستوية المتزايدة خطيا والكمون الكامل ( $ FP-LAP $ ) }

في طريقة $ LAPW-FP  $ الدالة الاساسية داخل كرة $ (T.M) $ تكون على شكل ترتيبات خطية للدالة الشعاعية $  U_{l} (r) Y_{lm} (r) $ وتمتاز باشتقاق $  U_{l}^{(1)} (r) Y_{lm} (r) $ بالنسبة للطاقة.
\begin{equation}\label{key}
	\phi(r) = \left\{\begin{array}{rcl} 
		\dfrac{1}{\sqrt\Omega} \sum C_{G} e^{i(G+K)r}    r & \leq & R_{\alpha}\\
		\sum \left\{A_{lm} U_{l}(r) + B_{lm} U_{l}^{(1)} (r)  \right\} Y_{lm} (r)    r & \geq & R_{\alpha}
	\end{array} \right.
\end{equation}
الدالة $ U_{l} $ تعرف مثل دالة الطريقة $ ( APW ) $ و الدالة $  U_{l} (r) Y_{lm} (r) $ تخضع للشرط التالي :
\begin{equation}\label{}
	\left\{\frac{d^{2}}{dr^{2}} + \dfrac{l(l+1)}{r^{2}} + V(r) - E_{l} \right\} r U_{l} (r) = 0 
\end{equation}
في الحالة اللانسبية الدوال $ U_{l} $ و $ U_{l}^{(1)} $ المستمرة دوما على سطح كرة $ (T.M) $ أي مستمرة مع الموجة المستوية في الخارج. إذن  الدالة $ APW  $ تصبح دالة أساسية للطريقة  $ LAPW $ أين المعاملات $ B_{lm} $ المكافئة للدالة $ U_{l}^{(1)} $ .\\
لها نفس طبيعة الدالة $ LAPWs  $ وهي الموجة المستوية الوحيدة في المنطقة الإقحامية . داخل الكرة الدالة $ LAPWs $ تعتمد على الدالة الدالة $ APWs  $ لأن  $ E_{l} $ تختلف قليلا عن عصابة الطاقة $ E_{g} $ . الترتيبات الخطية تنتج أحسن دالة شعاعية $ APWs $ معناه أن الدالة $ U_{l} $ يمكن أن تنشر على شكل الدالة المشتقة والطاقة $ E_{l} $ بالشكل :
\begin{equation}\label{}
	U_{l}(E,r) = U_{l}(E,r) + (E - E_{l}) U_{l} (E,r) + 0 (E - E_{l})^{2}
\end{equation}
حيث: $ 0((E - E_{l})^{2}) $ تمثل الخطأ الرباعي للطاقة . \cite{b5}

\subsection{ نظرية بلوخ ($ Bloch $) }
تعبر عن الخصائص الأساسية للبلورة والتي تتمثل في التناظر والدورية أين تكون الشوارد على شكل منظم وكمون البلورة $ (Vext(r $ متأثر بالإلكترونات التي تعبر عن الدورية والتي تتمثل عبارتها فيما يلي : 
\begin{equation}\label{key}
	 V_{ext}(\vec{r}) = V_{ext}(\vec{r} + \vec{R}) 
\end{equation}
بحيث :

	$ \vec{R} = l_{1}(\vec{a_{1}} + l_{2}(\vec{a_{2}} + l_{3}(\vec{a_{3}} )  $  هو شعاع يترجم شبكة مباشرة لـ $ Bravais $.\\


تعبر نظرية $ Bloc $ عن دالة الموجة لإلكترون أحادي لكوهن-شام $ \psi_{i} (\vec{r}) $  ومن الشكل الناتج لدالة موجة $ \exp (i\vec{k}\vec{r}) $   لدالة $ U_{i} (\vec{r}) $  التي تحتوي على التناظر والدورية في الشبكة البلورية $ M $ .
\begin{equation}\label{key}
	\psi_{i} (\vec{r}) =  U_{i} (\vec{r}) \exp (i\vec{k}\vec{r})
\end{equation}
حيث : $  U_{i} (\vec{r}) = U_{i}(\vec{r} + \vec{R})  $
\begin{itemize}
	\item 
	$ \vec{k}  $ : هو شعاع الموجة المحدودة في المنطقة الاولى لبريلون.
	\item 
	$ l $ : هو معامل القطاع.
		\item 
		$ \vec{R} $ : شعاع الشبكة المباشرة .
			\item 
			الدالة الدورية $ U_{i}(\vec{r}  $ يمكن نشرها باستعمال سلاسل فوري فنجد :
			\begin{equation}\label{key}
			U_{i}(\vec{r} = \sum C_{iG} \exp (i\vec{k}\vec{r}) 	
			\end{equation}
		\begin{itemize}
			\item 
			$ \vec{G} $ : هو شعاع الشبكة المعكوسة المعرفة ب $ G.R = 2mm $ حيث $ m $ عدد صحيح . 
			
			\begin{equation}\label{key}
					\psi_{i} (k , \vec{r}) = \sum C_{i , \vec{k}+\vec{G}} \exp (i\vec{k}+\vec{G}\vec{r}) 
			\end{equation}
						$ C_{i , \vec{k}+\vec{G}} $ :  تمثل معاملات النشر من اجل المدارات المشغولة.	
		\end{itemize}

\end{itemize}

تتعلق دوال الموجة بالنقاط $ K $ وهذا يؤكد القدرة على رسم دوال الموجة الالكترونية في حيز الفضاء $ K $ بدلالة الموجة بنقطة $ K $ ، نظرية $ Bloch $ تحدد الدراسة لدوال الموجة لوحدة الخلية الخاصة بالبلورة.
اذا جزء منها هو عبارة عن جزء محدود من الشبكة المعكوسة . \cite{b6}

\section{ برنامج $ Wien2k $ }

يعمل هذا البرنامج تحت نظام التشغيل $ LUNIX  $ حيث تم  تطوير برنامج المحاكاة  $ wien2k   $ في معهد كيمياء المواد بالجامعة التقنية في فينا وتم  نشره من طرف $ ( Blaha Schwarz,p ) $ في الحواسيب في السنوات التي مضت ذلك أجريت عدت تحديثات , وقد وضعت الإصدارات من برنامج $ wien  $ الأصلي ومنها ( $ wien97 , 95 wien, wein93 $ ) حيث سميت حسب سنة نشرها إلا أن نسخه  $ wien2k $ لعام  2000 شهدت استخداما كبيرا وهذا للتحسن الكيبر الذي عرفته لاسيما من حيث السرعة و سهولة الإستخدام .\cite{b1}

\subsection{ خوارزمية $ wien2k $ }

يتم العمل على $ Wien2k $ بإدخال معاملات البنية البلورية و مواقع الذرات في البلورة و نوعها ثم  نقوم بتحديد بعض الاختبارات على طريقة الحساب كشبه الكمون المستعمل و دقة الحساب , نشغل دورة $ SCF $ ونباشر في حساب الخصائص 
البنيوية و الالكترونية للمادة .\\
يمكن اضافة برامج مرفقة ل $ Wien2k $ كبرنامج  $ XceysDen  $ الذي يسمح بمشاهدة ثلاثية الابعاد لبنية المادة و الكثافة الإلكترونية وغيرها .
يقوم البرنامج برسم المنحنيات تلقائيا مع وضع البيانات اللازمة لذلك وإستنتاج بعض المعلومات الفيزيائية تلقائيا بفضل قاعدة بياناته التي تحتوي معلومات حول عناصر الجدول الدوري.\\
الأنظمة الإبتدائية للحساب : 
\begin{enumerate}
	\item 
	أبعاد الجوار الأقرب	$ NN $ : هذا البرنامج يستعمل الملف	$ case.struct $ والذي تكون فيه المواقع الذرية في خلية الوحدة المحددة ,من أجل حساب أبعاد الجوار الأقرب لكل الذرات ,ويتحقق من أّنها لا تتجاوز أنصاف الأقطار المرافقة إذا كان هناك تجاوز ,فإن رسالة تظهر على الشاشة خطأ. بالإضافة إلى هذا ,فإن أبعاد الجوار الأقرب الموالي الأعلى ب $ f $ مرة من بعد الجوار الأقرب $ f $ . لابد أن تكون محددة تلقائيا تكتب في ملف المخرجات الذي يسمى  $ outputnn.case  $ قطر المجال الذري يسمح بإضافة ملف التحكم $ ( case.struct ) $ وملف الإخراج لهذا يسمى  $ (outputnn.case) $ .
	\item 
	$ SGROUP $  :  
	يحدد مجموعات الفضاء للبنية المعرفة في الملف ( $ Struct. cas $ ) الذي تكون فيه معلومات عن ( نوع الشبكة ,ثابت خلية الذرات المركبة ) ويحدد فضاء المجموعات ويستخدم الشحنات النووية لإنتاج الإختلافات للذرات الإستثنائية , 
	كما يمكنه إيجاد أصغر خلية للوحدة , ويجعل ملف الإخراج هو $ sgroup $ $ struct.case  $ إعتماد على الملف السابق $ struct.case  $ , ينفذ هذا البرنامج على الأوامر التالية :
$ 	Sgroup -wi case.struct [-wo case.struct sgroup] case.outputsgen $
	\item 
	التماثل $ SYMMETRY $ : هو البرنامج الذي يسمح بحساب عمليات التناظر للمجموع الفضائي حيث المعلومة الواردة تكون في الملف ( $ struct.cas $ ) الذي يعطي معلومات عن نوع الشبكة , مواضع الذرات ...إلخ . كما يمكن أن نحدد من 
	خلاله المواقع النووية المختلفة و مصفوفة التناوب المقابلة .
	برنامج $ LSTART $ :  ينتج الكثافة الإلكترونية للذرات الحرة و يحدد كيفية التعامل مع المدارات المختلفة  في عصابة الطاقة . 
	حيث أن هذا النظام ينتج الكثافات الذرية التي يستخدمها $ dstart  $ حتى يجد كثافة الحالات الذرية الاولية من خلال الحسابات .
	\item 
	$ Kgen $ : يولد شبكة في نقطة $ K $ غير القابل للإختزال من للاختزال المنطقة بريليوين الأولي ($ Z.B $) وهي تحدد عدد العناصر في جميع أنحاء $ ZB $ $ K $الأولى
	\item 
	$ Dstart $ : يتكون من خمسة برامج فرعية يصدر الكثافة الأولية لحلقة $ SCF $ من خلال تراكب كثافة ذرية تنتج
	
	في $ Lstart. $ كما يسمح بتهيئة حساب تم إنشاؤه بإستخدام كل القوائم لحلقة SCF ثم تبدأ العملية مع التكرار حتى ملاقاة
	الحل هذة الحلقة يتم إستدعاؤها بواس   $ [Blaha P. et al 2001], run_lapt  $ .
\begin{list}{}{}
	\item 
	 بحسب الكمون الكلي انطلاقا من ا الكثافة $ LAPWO $ .
	\item
	 يحسب عصابات التكافؤ, المقدار الخاص والشعاع الخاص $ LAPW1 $ .
	\item
	 يستخدم ملف $ cas.vector $ ليحسب طاقة فرمي, توسعات الكثافة الإلكترونية للتكافؤ $ LAPW2 $ .
	\item
	 يحسب الحالات القلبية للكمون في الجزء الكروي $  LCORE $ .
	\item
	 برنامج فرعي يستخدم الكثافة الإلكترونية القلبية, الحالات النصف قلبية وحالات التكافؤ تخلط لإنتاج الكثافة كلية جديدة تستخدم في التكرار $ MIXER $ .\cite{b4}
\end{list}

\end{enumerate}

\subsection{  استخدامات برنامج $ wien2k $ }

يتكون هذا البرنامج من عدة  برامج مستقلة لإجراء عمليات حسابية للبنية الإلكترونية في المواد الصلبة وهذا اعتمادا على نظرية الكثافة الوظيفة ( $ DFT $ ) فيمكن من خلاله حساب  :
\begin{list}{}{}
	\item 
	عصابات الطاقة وكثافة الدوال لسطح فيرمي .
	\item 
	الكثافة الإلكرونية وكثافة السبين وعوامل البنية للأشعة السينية .
	\item 
	إستقطاب السبين ( في حالة ما إذا كانت البنية تتعلق بالعازل الكهربائي الشفاف ) .
	\item 
	تدرج الحقل الكهربائي .
	\item 
	الطاقة اإلمجالية , القوى النووية , هندسة توازن الذرات في الفضاء ( التحسينات البنيوية ) .
	\item 
	الخصائص البصرية .
	\item 
	إنبعاثات وإمتصاصات الأشعة السينية $ RX $ .
\end{list}

\begin{figure}[h!]
	\centering
	\includegraphics[width=0.7\linewidth, height=0.7\textheight]{Fig/Fig_II/wien2k}
	\caption{مخطط عام لبرنامج wien2k  }
	\label{fig:wien2k}
\end{figure}
\FloatBarrier

بداية يتم تشغيل خادم $ w2web $ ثم نتبع الخطوات الآتية :
\begin{enumerate}
	\item 
	الاتصال بخادم $ w2web $ : استخدم متصفح $ WWW $ المفضل لديك للاتصال بـ $ w2web $ ، مع تحديد رقم المنفذ الصحيح ، على سبيل المثال : \\
	$ netscape http://hostname_where_w2web_runs:7890 $
	
	\item 
	إنشاء حالة جديدة : نستخدم واجهة المستخدم $ w2web $ سلاسل للتمييز بين بيئات العمل المختلفة وللتغيير السريع بين الحسابات المختلفة. عليك أولاً إنشاء حالة جديدة  (أو اختيار فصل قديم). أدخل " $ TiC $ " وانقر فوق الزر "  إنشاء " .\\
	ملاحظة: لا يؤدي إنشاء حالة إلى إنشاء دليل جديد تلقائيًا !	\\
	سيتم تحويل ذلك في الدليل الرئيسي الخاص بالمستخدم إذا لم يتم تعيين دليل عمل لهذه الحالة مسبقًا (أو إذا لم يعد الدليل موجودًا) كما هو موضح في الشكل .
	
	\begin{figure}[h!]
		\centering
		\includegraphics[width=0.8\linewidth, height=0.4\textheight]{Fig/Fig_II/w2web}
		\caption{ صورة شاشة لإعدادت البدء على المستخدم $ w2web $  }
		\label{fig:w2web}
	\end{figure}
	\FloatBarrier
	
	\item 
	إنشاء دليل حالة جديد : باستخدام "  $ 	Session Mgmt.--> change directory    $  " يمكنك تحديد دليل موجود أو إنشاء دليل جديد. في هذا المثال تم إنشاء دليل جديد $ lapw $ لمركب $ TiC  $ باستخدام الزر " إنشاء " ثم النقر فوق تحديد الدليل الحالي يتعين هذا الدليل الذي تم إنشاؤه حديثًا إلى الحالة الحالية .
	
	\begin{figure}[h!]
		\centering
		\includegraphics[width=0.8\linewidth, height=0.4\textheight]{Fig/Fig_II/w2k}
		\caption{ 	الواجهة الرئيسية للمستخدم $ w2web $ على برنامج $ wien2k $ }
		\label{fig:w2k}
	\end{figure}
	\FloatBarrier

	\item 
	إنشاء ملف الإدخال الرئيسي " file.input " : لإنشاء الملف $ TiCstruct $ ، البدأ بإنشاء ملف هيكلي باستخدام " $ Execution --> StructGen $ "  .\\
	بالنسبة للحالة الجديدة ، يقوم $ w2web $ بإنشاء قالب هيكل فارغ يمكن من خلاله تحديد البيانات الهيكلية و ملء البيانات الواردة أدناه في الحقول المقابلة ( المربعات البيضاء ) :
	
	\begin{figure}[h!]
		\centering
		\includegraphics[width=0.8\linewidth, height=0.4\textheight]{Fig/Fig_II/w2k1}
		\caption{  صورة شاشة للبيانات الهيكلية للمركب }
		\label{fig:w2k1}
	\end{figure}
	\FloatBarrier
	
	\item 
	بدء الحساب بعملية التهيئة  ( $  init\_lapw $ ) : بعد إنشاء ملفي الإدخال الأساسيين ، يتم تفعيل الحساب بواسطة "  $ Execution  initialize.calc $  " . سيرشد هذا خلال الخطوات اللازمة لبدء الحساب. ما بعد ذلك اتباع الخطوات الموضحة باللون الأخضر واتباع التعليمات أو بدلاً من ذلك ، يمكنك تشغيل البرنامج النصي $ init_lapw  $ من سطر الأوامر ثم تسجيل جميع إجراءات هذا البرنامج النصي باختصار .

	\item 
	حساب SCF : بعد إعداد الحالة ، يتم تشغيل البرنامج النصي $ run_lapw $  ثم استدعاء دورة التناسق الذاتي $ SCF $ باستخدام $ run SCF $ .
	
	حيث تمر دورة $ SCF $ عبر الخطوات التالية :
	\begin{list}{}{}
		\item 
		$ LAPW 0 $ 	يولد الإمكانات من الكثافة (المحتملة) . 
						 
		\item
		$ LAPW 1 $ يحسب نطاقات التكافؤ او ما يسمى بالعصابات (القيم الذاتية والمتجهات الذاتية) .

		\item
		$ LAPW 2 $  ($ RHO $)  .يحسب كثافة التكافؤ من المتجهات الذاتية									
		\item
		$ LCORE $				.	يحسب الحالات الأساسية والكثافات
		\item 
		$ MIXER $				.	يمزج بين كثافات المدخلات والمخرجات
		
	\end{list}
	
	لتشغيل دورة $ SCF $ ، انقر فوق " تشغيل ! " نظرًا لأن هذا قد يستغرق وقتًا طويلاً للأنظمة الأكبر ؛ يمكنك تحديد " نوع التنفيذ " ليكون دفعة أو إرسال إذا تم تكوين النظام باستخدام نظام قائمة انتظار و يتم إعداد $ w2web $ بشكل صحيح .	
	أثناء تشغيل دورة $ SCF $ لـمركب معين ، حاول مراقبة على سبيل المثال إجمالي الطاقة (التسمية: $ ENE $) أو مسافة الشحن (التسمية: $ DIS $ ). قد يتقارب الحساب ، عندما يتم استيفاء معيار التقارب لثلاثة تكرارات لاحقة ( ثم تتم مقارنة مسافة الشحن في المثال ) .
	
	كما يمكننا ايضا تحسين حساباتنا بالتحقق من تقارب أهم المعلمات و تطبيق طريقة mBJ عليها .
	\item 
	حساب الخصائص : بمجرد تقارب دورة $ SCF $ ، يمكن للمرء حساب خصائص مختلفة مثل كثافة الدالة الكلية ($ DOS $) و حتى الجزئية  أو بنية النطاق أو الخصائص البصرية أو الالكترونية .\\
	لحساب الخصائص من الافضل استخدام واجهة المستخدم ، $ w2web $. حيث تتوفر على قوالب ملفات الإدخال تلقائيًا وتوضح كيفية حساب الخصائص خطوة بخطوة .
	=======
	 صورة 
	 ====
	 
\end{enumerate}

\section*{الخاتمة}
