
\chapter{ أنصاف النواقل }

\label{Chapter1}
\section*{مقدمة}

‫ﺗﻮﺟﺪ ‫اﳌﺎدة‬  ‬في  ‫ﺣﺎﻻت‬ ‫أرﺑﻌﺔ‬ هي : الحالة الصلبة و السائلة و الغازية أو البلازما , و يختلف ذلك من حالة لأخرى حسب البنية البلورية للمادة و قوى الربط بين جزيئاتها و تقبل كل حالة التحول  إلى أخرى تحت تأثير عوامل فيزيائية مختلفة ( P , T , V … ) .
كما تصنف المادة وفقا لخصائصها الكهربائية الى ثلاث , النواقل و العوازل و أشباه النواقل حيث تشكل هذه الاخيرة قاعدة أساسية للبنى التحتية الالكترنية و التى اصبح العالم اليوم يطالب بالاطلاع على  تعريفاتها, مبادئها و تطبيقاتها\cite{a2}.

\section{مبادئ و تعريفات}

\subsection{البنية البلورية }

\subsection*{تعريفها :}
\subsubsection*{لغـة : }
\begin{list}{}{}
	\item 
	بِنية : (اسم) ; الجمع : بِنًى و البنى في اللغة هي: التركيب و القوام .
\item	
بلورية : صفة لإسم التبلور و هي مشتقة من كلمة بلورة و هي : تموضع او تكون أجسام متشاكلة تتبلور معًا في بلّور واحد متجانس . \cite{a4}
\end{list}

\subsubsection*{فيزيائيا :}
البنیة البلوریة هي تجمع عدد لانهائي من الوحدات الذرية المتماثلة تتكرر بشكل دوري منتظم (اذا كانت مثالية) في كل إتجاهات الفضاء , حيث تتميز باستقرارها غالبا وامتلاكها لكثافة أصغرية وهذا التميز يحقق المعالم التالية :
\begin{enumerate}
	\item
	تحافظ على الاعتدال الكهربائي في البلورة .
	\item
	تكون جميع الروابط بين الذرات محددة .
	\item
	تجتمع الذرات معا لتحتل حجما أصغريا . 
\end{enumerate}

\begin{figure}[bh]
	\centering
	\includegraphics[width=0.7\linewidth]{Fig/Fig_I/Three-solid-polymorphs-of-GaN-a-wurtzite-b-zinc-blende-and-c-rock-salt-Ga-and-N.ppm}
	\caption{مثال عن البنيات البلورية الممكنة لمركب GaN }
	\label{fig:three-solid-polymorphs-of-gan-a-wurtzite-b-zinc-blende-and-c-rock-salt-ga-and-n}
\end{figure}
\FloatBarrier

\subsubsection{النظم البلورية و شبيكات برافيه :}
تنسب إلى عالم البلورات الفرنسي برافيه (Bravais) حيث شبكة تبلور برافيه في الهندسة وعلم البلورات هو مجموعة نقاط منتظمة في الفراغ لا نهائية، يسهل وصفها عن طريق مسافات بينية متساوية أو إزاحات متماثلة في الطول وزاوية الازاحة. 

\begin{figure}[bh]
	\centering
	\includegraphics[width=0.8\linewidth]{Fig/Fig_I/chabakatBravais}
	\caption{طرق تعيين شبيكة برافييه}
	\label{fig:chabakatbravais}
\end{figure}
\FloatBarrier

تصنف الشبيكات البلورية إلى أربعة عشرة شبيكة موزعة على سبعة أنظمة بلورية $ Systems Crystal $ محدودة بعدد الطرق الممكنة لترتيب النقاط .الشبيكة وعدد شبيكات برافية أربعة عشر بحيث تكون البيئة المحيطة بأي نقطة منها مماثلة تماما للبيئة المحيطة بأي نقطة أخرى.
النظام البلوري شبيكات برافيه خصائص خلية الوحدة : \cite{a1}

\begin{figure}[th]
	\centering
	\includegraphics[width=0.8\linewidth]{Fig/Fig_I/bravaisGaN}
	\caption{الأنظمة الأربعة عشر لخلايا برافيه }
	\label{fig:bravaisgan}
\end{figure}
\FloatBarrier

\subsection{الخصائص الفيزيائية البلورية }
إن الخاصية الفيزيائية عبارة عن أي خاصية للمادة تكون قابلة للقياس، وبمعرفة مقدار قيمتها، يمكن وصف حالة المادة أو النظام الفيزيائي للمادة في لحظة زمنية محددة، كما تصنف الخواص الفيزيائية بدورها إلى : 

\subsubsection{الخصائص الضوئية :  }
	دراسة الخواص البصرية لأشباه الموصلات تعطي إشارة واضحة في تحديد طبيعة استخدامها، ولفهم آلية الانتقالات الالكترونية بين حزم الطاقة من خلال قياس امتصاص الأشعة للمادة شبه الموصلة و نفاذيتها، وأن جميع المواد شبه الموصلة تشترك بصفة مهمة ومميزة في مخطط الامتصاصية ، وهي مساوية تقريبا ً إلى طاقة الفجوة ،التي تفصل بين حزمتي التكافؤ والتوصيل والتي تدعى بحافة الامتصاص الأساسية $ ( Fundamental Absorption Edge )  $ , وتتمثل هذه الخصائص في ثلاث معاملات أساسية هي : 
\subsubsection*{النفاذية $ Transmittance(T)  $ :  }
		 تعرف بأنها النسبة بين شدة الإشعاع النافذ عبر الغشاء(IR TR)إلى الشدة الأصلية للإشعاع الساقط عليه(IR oR)، وهي أيضا ً كمية خالية من الوحدات، وتعطى بالعلاقة الآتية :  \begin{equation*} 	T = \dfrac{I_{T} }{I_{0}}\end{equation*}
		
		\subsubsection*{الامتصاصية $ Absorptance (A) $ : }
		 تعرف بأنها النسبة بين شدة الإشعاع الممتص الذي يمتصه الغشاء(I A ) إلى الشدة الأصلية للإشعاع الساقط عليه (I R ) وتكون الامتصاصية كمية خالية من الوحدات، وتعطى بالعلاقة الآتية : \begin{equation*} 	A = \dfrac{I_{A} }{I_{0}}\end{equation*}
		
				\subsubsection*{الانعكاسية $ (R) Reflection $ :} 
		 تعرف بأنها النسبة بين شدة الإشعاع المنعكس عن الغشاء باتجاه معين إلى الشدة الأصلية للإشعاع الساقط عليه، وتعطى في المعادلة الآتية : \begin{equation*} 	R = \dfrac{I_{R} }{I_{0}}\end{equation*}
		وترتبط الامتصاصية(A)بالانعكاسية(R)والنفاذية(T)كما في العلاقة الآتية : A + R + T = 1
	
	\subsubsection{الخصائص الإلكترونية : }
	
	 ترتبط معظم الخصائص الفيزيائية للمادة الصلبة بالخصائص الإلكترونية التي تكمن في تحديد السلوك الإلكتروني وفهم الروابط بين الذرات المكونة للمادة وطاقة الفجوة.وتوضح هذه الخواص عن طريق بنية العصابات وكثافة الحالة و الشحنة ، و نوضح هذه الخصائص فيما يلي :
	
			\subsubsection*{عصابات الطاقة :}
		 تعتمد نظرية عصابات الطاقة على تشكيل الطاقة مع اخذ الإلكترونات الداخلية لمعادلة الإنتشار ( E ) k التي من خلالها يمكن إيجاد قيم مهمة أهمها الطاقة Eg ( الفجوة الطاقوية ) , الكتلة الفعالة, عرض عصابة التكافؤ و انتقال الإلكترونات . 
		بالإستعانة بقيمة الفاصل الطاقوي Eg يمكن معرفة نوعية المواد إما عازلة ،معدن ،شبه ناقل ،أونصف معدن. و التبرير بدراسة كثافة الحالة و شحنتها .\\
		
		\begin{figure}[th]
			\centering
			\includegraphics[width=0.8\linewidth]{Fig/Fig_I/gap}
			\caption{بنية عصابة الطاقة}
			\label{fig:gap}
		\end{figure}
		\FloatBarrier
		
		فالمادة الصلبة هي نظام معقد يحتوي على عدد كبير من الذرات والإلكترونات حيث تحتوي الذرة على عدة مستويات طاقة متميزة، والإلكترون المرتبط بهذه الذرة موجود بالضرورة فيأخذ هذه المستويات. ومع ذلك، في التركيب البلوري تتجمع مستويات الطاقة للذرات المستقلة معًا في البنية البلورية الكلية لتشكيل حزم طاقة، التي هي مفصولة بمناطق "محظورة "، يوجد رسم توضيحي في الشكل الموالي يعرض مستويات الطاقة والعصابات في مخطط عمودي، فبعد زيادة الطاقات أعلى حزم الطاقة هما حزمة التكافؤ وحزمة التوصيل مفصولة بما يسمى " فجوة طاقة " . ومن أجل التبسيط ، تتم معالجة نقاط عالية التناظر  فقط ( K point ) في منطقة Brillouin الأولى.\\ \cite{a3}
		
		\begin{figure}[h]
			\centering
			\includegraphics[width=0.8\linewidth]{"Fig/Fig_I/Sans titre8"}
			\caption{منطقة بريلوان}
			\label{fig:sans-titre8}
		\end{figure}
		\FloatBarrier
		
		نقوم بدراسة تغيرات الطاقة للإلكترون بدلالة الشعاع الموجي K في الفضاء المعكوس ، لإيجاد المانع الطاقي الذي يمثل الفرق بين القيمة الحدية العظمى لعصابة التكافؤ والقيمة الحدية الصغرى لعصابة النقل في نفس النقطة أو نقطتين مختلفتين. و من هنا نميز نوعين من الفجوات الطاقوية :
		
		\begin{enumerate}
			
			\item 
			فجوة طاقوية مباشرة ( Gap direct ) عند انتقال الإلكترون من قمة حزمة التكافؤ إلى قعر حزمة التوصيل عند النقطة نفسها بصورة عمودية، يسمى هذا الإنتقال بالإنتقال المباشر و حينئذ تكون الفجوة مباشرة .
			
			\item 
			فجوة طاقوية غير مباشرة ( Gap indirect ) عند انتقال الإلكترون من أعلى قمة في حزمة التكافؤ إلى أدنى قمة في حزمة التوصيل بصورة غير عمودية, يسمى عندها هذا الإنتقال بالإنتقال غير المباشر و حينئذ تكون الفجوة غير مباشرة .
			
			\begin{figure}[h]
				\centering
				\includegraphics[width=0.8\linewidth]{Fig/Fig_I/bandgap1}
				\caption{الانتقالات الالكترونية المباشرة و الغير مباشرة}
				\label{fig:bandgap1}
			\end{figure}
		\FloatBarrier
			
		\end{enumerate}
		
					\subsubsection*{كثافة الشحنة :}
		 و هي التي من خلالها يمكن معرفة الروابط حيث نأخذ المستوي الذي يشمل الذرة التي احداثيها ( 0 , 0 ,0 ) ، ( 0.25 , 0.25 , 0.25 ) و المحدد بالشعاعين ( 0 , 1 , 1 ) و ( 1 , 0 , 0 )  على سبيل المثال .
		
							\subsubsection*{كثافة الحالة :}
		 تعرف كثافة الحالات بعدد الحالات الالكترونية لطاقة معينة, فهي تسمح بتحديد خصائص التوصيل الالكتروني للمادة كما انه من اجل كل ذرة نعرف كرة ذات قطر في الداخل بحيث نسقط الكثافة الالكترونية على التوافقية الكروية من نوع s,p,d فنحصل كذلك على كثافة الحالة الجزئية التي تسمح بتحديد بنية الروابط الكيميائية بين ذرات البلورة او الجزيء .\\
		تعتمد إسقاطات كثافة الحالة الإجمالية على نصف قطر الكرة التي تسقط عليها كثافة الحالة الجزئية , و يمكن حسابها بطريقة التدرج المعمم .
	
		\subsubsection{الخصائص المغناطيسية : }
		\subsubsection*{السبين الالكتروني : }
	 
	في السنوات الأخيرة ظهر في مجال التكنولوجي مصطلح السبين الالكتروني، الذي لا يدرس شحنة الإلكترون فقط ولكن يهتم بمجال المعلومات الناتجة عن سبين الإلكترونات .
	
بحيث يكشف علاقة التفاعل بين الكترونات النقل و الخصائص المغناطيسية المواد على مستوى الميكروالكتروني ، أي يضم درجة حرية السبين للقاعدة التقليدية لالكترونات أنصاف النواقل، و يظهر ذلك في الدمج بين المجال الالكتروني، الضوئي و المغناطيسي ( أي السين متعدد الدوال)، وكمثال على ذلك حقيقة حقل ترتمیسور $ spin-FET $ مصدر ضوء الديود $ spm-LED $ ، جهار التجارب التنفي	$ spin RTD  $ ، جهاز تحويل الرسالة الى رموز $ encoders $ جهاز حل الشهرة $ Decoden $ ، وسائل الاتصال عموما، وفي أجهزة حفظ المعلومات .
واحدة من العوائق التقنية للأجهزة السابقة هو عملية التصنيع الجسيمي لأنصاف النواقل بقاعدة سيين الكتروني، البحث في المواد المغناطيسية و انصاف النواقل يتطلب تحديات كثيرة للاختلاف الموجود في البني البلورية والروابط الكيمائية ، فكثير من أنصاف النواقل المغناطيسية صنعت عن طريق التطعيم، ومن العوائق بعد التصنيع هو درجة الحرارة لإمكانية العبور من حالة المادة المغناطيسية إلى الحالة اللامغناطيسية و المعروفة بدرجة حرارة كوري  .
	ان كثير من المواد الموجودة في الطبيعة هي مركبات لامغناطيسية و عند تطبيق حقل مغناطيسي خارجي يكتسب بعضها خاصية التمغنط ، على المستوى الالكتروني تصطف الالكترونات باتجاه الحقل المغناطيسي الخارجي فترتب على ثلاثة طرق هي :
	\begin{enumerate}
		\item 
		الطريقة الأولى تسمى الحالة المغناطيسية الحديدية وتكون جبهة الإلكترونات كلها في حيه العمل وسبين أعلى .
				\item 
		الطريقة الثانية تسمى الحالة ضد المغناطيسية الحديدية وتكون جهة الالكترونات في جهة وعكس جهة الحقل التداول .
		\item
		الطريقة الثالثة تسمى بالحالة اللامغناطيسية حيث تكون فيها حركة الإلكترونات حركة عشوائية .
	\end{enumerate}
	 
	كما يجدر بالذكر أن كل الحالات السابقة يمكنها العبور من الحالة المغناطيسية و الضد مغناطيسية إلى الحالة اللامغناطيسية إذا استخدم معامل درجة الحرارة .
	
	\begin{figure}[h!]
		\centering
		\includegraphics[width=0.8\linewidth, height=0.4\textheight]{Fig/Fig_I/mgntq3}
		\caption{حالات تواجد سبين إلكترونات المواد}
		\label{fig:mgntq3}
	\end{figure}
\FloatBarrier

		\subsubsection*{العزم المغناطيسي : }
		 ==========
		 \\
		 =====================
		 
\paragraph{أهمية معرفة الخصائص الفيزيائية }

تعد دراسة الخواص الفيزيائية لشبه الموصل من المصادر المهمة جدا في اعطاء معلومات حول حزمته الالكترونية, فجوة الطاقة وتردد الأنماط البصرية والسماحية و مجموعة النتائج التي تبين للباحث  تركيبه البلوري بالإضافة الى بعض العيوب و التشوهات التي يتعرض اليها شبه الموصل .مما يسهل التطبيقات الصناعية و المختبرية على المواد الشبه موصلة.

\subsection{تصنيف المواد الصلبة حسب ناقليتها }
تصنف المواد الصلبة في الطبيعة حسب توصيليتها الكهربائية عند
درجة حرارة الغرفة إلى مواد موصلة وهي ذات توصيلية كهربائية عالية و مواد عازلة أوطأ منها بكثير و مواد شبه ناقلة توصيلتها تقع بين المواد الموصلة والعازلة .\\
أعتمد التصنيف على اساس تركيب الحزم $ ( Band ) $ للمادة وعلى مقدار فجوة الطاقة التي تفصل حزمة التوصيل $ (conduction) $ عن حزمة التكافؤ $ (valance) $ حيث تكون كبيرة في العوازل وأقل في أشباه النواقل ومنعدمة في النواقل.\\
يتطلب التوصيل الكهربائي انتقال الالكترون من حزمة التكافؤ المملوءة بالإلكترونات الى حزمة التوصيل الفارغة من الالكترونات عبر الفجوة المحظورة بينهما اي اكتساب الالكترون طاقة للانتقال من حزمة الى حزمة ويطلق على هذه الطاقة $ (Eg) $ بفجوة الطاقة أما بالنسبة الى المواد شبه الناقلة فإن الفرق الاساسي بينها وبين المواد العازلة يكمن في قيمة فجوة الطاقة التي تكون أقل بكثير من قيمة فجوة الطاقة في المواد العازلة, بحيث  في المواد العازلة $ (  Insulator ) $ تكون فجوة الطاقة الممنوعة كبيرة بحدود $ eVP (5-10)  $ و في أشباه النواقل $ eV 0.5-3.5  = E{g} $ أما عن النواقل فهي أفرطهم .
\begin{figure}[h]
	\centering
	\includegraphics[width=0.8\linewidth]{"Fig/Fig_I/Sans titre10"}
	\caption{مخطط حزم الطاقة للمواد الناقلة ,العازلة و أنصاف النواقل }
	\label{fig:sans-titre10}
\end{figure}
\FloatBarrier

\section{عموميات حول أنصاف النواقل}
\subsection{تعريفها }
\paragraph{لغـــة :}
أَنْصَاف: (اسم)
أَنْصَاف : جمع نِصف و النِّصْفُ في اللغة هو: شَطْرُ الشيءِ 
نواقِلُ : (اسم)
نواقِلُ : جمع ناقِل و الناقِل في اللغة هو: حامل الشيء
و بالتالي أنصاف النواقل في اللغة هي حاملات لنصف الأشياء .
و أنصاف النواقل الكهربائية هي حاملات نصف الشحنة . \cite{a1}

\paragraph{فيزيائيــا :}
أنصاف النواقل هي مواد تتوفر على الأرض إما في شكل عناصر فيزيائية خالصة تقع في العمود الرابع من الجدول الدوري وهي عنصري الجرمانيوم و السليسيوم، أو من مواد مركبة ناتجة عن خلط بعض عناصر العمود الثالث كالبور والألمنيوم والأنديوم والقاليوم مع عناصر و العمود الخامس كالفوسفور والزرنيخ 
(الأرسنيد) منتجة مواد نصف ناقلة كفوسفيد الأنديوم وأرسنيد الغاليوم او نتريدات الغاليوم وغيرها من المركبات التي قد تتفوق على العناصر النصف ناقلة الخالصة في بعض خصائصها الكهربائية.\\
تقع بين النواقل والعوازل من حيث الناقلية الكهربائية،تتميز بوجود ثلاث عصابات وهي عصابة النقل و عصابة التكافؤ بينهما عصابة ممنوعة،أهم هذه المواد السليسيوم والجرمانيوم و أضيفت لها مؤخرا عناصر الارض الناذرة اكتشافا بأنها اكثر فعالية. للطاقة الحرارية دورا ً مهماً في مساعدة الالكترونات لعبور $ Eg  $ , فناقليته تتأثر بـعوامل فيزيائية عديدة كـ $ ( T,B,c,n ) $ فحساسية شبه الموصل تجاه هذه العوامل تجعل منه مادة بالغة الاهمية في التطبيقات الالكترونية .

\subsection{نبذة تاريخية عنها }

وفقا لجي.بوش   استخدم مصطلح "أشباه الموصلات" لأول مرة من قبل أليساندرو فولتا عام 1782 .
أول ملاحظة موثقة لتأثير أشباه الموصلات كانت من العالم مايكل فاراداي (1833) ، الذي لاحظ ان مقاومة كبريتات الفضة تنخفض مع زيادة درجة الحرارة  والتي كانت مختلفة عن الاعتماد  الملحوظ في المعادن .
كما وجد تحليل كمي شامل لدرجة الحرارة متعلقة بالموصلية الكهربائية لـ Ag 2 S و Cu 2 S حيث تم نشره عام 1851 من طرف يوهان هيتورف . بعدها ببضع سنوات جاء تاريخ أشباه الموصلات مركزا على خاصيتين مهمتين i.e.\\ \cite{a5}
استخدم الصمام الالكتروني في عملية التضخيم الهاتف ولكنه كان غير موثوق بالاضافة الى استهلاكه العالي للطاقة و لاشعاعه الحراري قفي الثلاثينيات من القرن العشرين توضح للمسؤول عن مختبرات پل مارقين كيلي بان هناك حاجة ماسة لجهاز جديد يواكب تطلعات شركة الهاتف راغبة في مزيد من التوسع. بدأ العمل على صنف ذي مواصفات غريبة من العناصر سمي پاشباه الموصلات. خلال الحرب العالمية الثانية عملت مختبرات پل بشكل حثيث لتطوير بلورة نقية من عنصر شبه موصل هو الجرمانيوم لاستخدامها في مازج الترددات المستعمل في وحدات استقبال الرادارات. كما نجح نفس المخبريون وقتها في انتاج بلورات اشباه موصلات من الجرمانيوم عالية الجودة .
في أواخر سنة 1947 نجح والتر براتين و جون باردين و ووليام شوكلي في مختبرات پل استنادا على ما سبق في صناعة أول ترانزيستور باستعمال الجرمانيوم.
و تواصل اجتهادهم في تحسينها الى عام 1956 أين حصل المخترعون الثلاثة على جائزة نوبل في الفيزياء نتيجة أبحاثهم في أشباه النواقل و اكتشاف تأثير الترانزيستور. \cite{a6}

\begin{figure}[bh]
	\centering
	\includegraphics[width=0.8\linewidth]{Fig/Fig_I/IMG_20220419_150942}
	\caption{العلماء المشاركين في اكتشاف أشباه الموصلات}
	\label{fig:img20220419150942}
\end{figure}
\FloatBarrier

\subsection{ أنواعها}

\begin{enumerate}
	\item 
	نصف ناقل ذاتي : نقول عن نصف ناقل ذاتي إذا كان عدد الإلكترونات يساوي عدد الفجوات أي أن كثافة الإلكترونات في 
	حزمة التوصيل تساوي كثافة الفجوات في حزمة التكافؤ .
	\item 
	نصف ناقل الغير ذاتي (مطعم/مشوب ): نقول عن نصف ناقل أنه غير ذاتي إذا أضفنا له بعض الشوائب فتدخل ضمن تركيب البنية البلورية لشبه الموصل الذاتي فتحوله إلى شبه موصل مشوب. 
	و هناك نوعين هما حجر الاساس الذي يستخدم في بناء العديد من العناصر الإلكترونية، و التي تقوم عليها صناعة الإلكترونيات كامًلة, نذكرهما :
	
	\subsubsection*{ النوع آن $ N $ :}
	
		 إذا كان عدد إلكترونات التكافؤ في المادة الشائبة أكبر من عدد إلكترونات التكافؤ فيشبه الموصل الاذاتي، فسيكون هناك فائض من الإلكترونات في البنية البلورية الجديدة، ويوصف شبه الموصل عندهابأنه من النوع $ N $ ، و هذا الحرف اللاتيني نسبة لكلمة $ Negative $ ، وذلك إشارة إلى شحنة الإلكترونات السالبة.
		
		\begin{figure}[h]
			\centering
			\includegraphics[width=0.7\linewidth]{"Fig/Fig_I/Sans titre11"}
			\caption{ شبه موصل من نوع آن $ N $ }
			\label{fig:sans-titre11}
		\end{figure}
	\FloatBarrier
		 
		مثال : $ Si $ أو $ Ge $ الذين يتبلوران في بنية ماسية، حيث تكون كل ذرة مربوطة بأربع ذرات قريبة بروابط تكافؤية. ندخل ذرة تحتوي على خمس الكترونات تكافؤ $ (P،As...) $ ، لكي نعوض ذرة من البلورة. من بين الالكترونات الشوائب أربعة تشارك في الروابط التكافؤية مع الأقارب والخامسة تبقى منفردة ، ذرة الشوائب مربوطة بمستوي يسمى المستوى المعطي الموجود تحت عصابة التوصيل، في هذه الحالة نسمي نصف الناقل من نوع $ N $ .
		
		\begin{figure}[h]
			\centering
			\includegraphics[width=0.7\linewidth]{"Fig/Fig_I/Sans titre4"}
			\caption{البنية الجزيئية لشبه موصل من نوع آن $ N $}
			\label{fig:sans-titre4}
		\end{figure}
	\FloatBarrier
		
	\subsubsection*{ النوع بي $ P $ :}
	
		 إذا كان عدد إلكترونات التكافؤ في المادة الشائبة أقل من عدد إلكترونات التكافؤ في شبه الموصل اللاذاتي، فسيكون هناك فائض في الثغرات الإلكترونية. ويوصف شبه الموصل الناتج بأنه من النوع بي  $ P $ ،وهذا الحرف اللاتيني مأخوذ من الحرف الاول لكلمة موجب $ Positive $ ، وذلك إشارة إلى شحنة الثغرات الإلكترونية الموجبة (الفجوات) . \cite{a19}
		
		\begin{figure}[bh]
			\centering
			\includegraphics[width=0.6\linewidth, height=0.3\textheight]{"Fig/Fig_I/Sans titre12"}
			\caption{ 	شبه موصل من نوع بي $ P $  }
			\label{fig:sans-titre12}
		\end{figure}
		\FloatBarrier

		مثال : إذا أدخلنا ذرة لها ثلاث الكترونات تكافؤ $ (B،Al ...) $ هذه الذرة لا يمكنها تشكيل سوى ثلاث روابط تكافؤية ، إذن تبقى رابطة تكافؤية لذرة التشويب غائبة ومتعلقة بمستوى الطاقة الواقع فوق عصابة التكافؤ، هذا المستوى يسمى المستوى المستقبل في هذه الحالة نسمى نصف ناقل من النوع بي $ P $ .
		\begin{figure*}[h!]
			\centering
			\includegraphics[width=0.6\linewidth, height=0.3\textheight]{"Fig/Fig_I/Sans titre6"}
			\caption{البنية الجزيئية لشبه موصل من نوع بي}
			\label{fig:sans-titre6}
		\end{figure*}
		\FloatBarrier
	
\end{enumerate}

\subsection{ تطبيقاتها}

لقد تم استخدام المواد النصف ناقلة في صناعة الترانزستور و النبائط ذات الطرفين كالمقومات والثنائيات الضوئية وغيرها. لما تتميز به من خصائص فريدة عند موصيليتها، خصوصا بعد إضافة شوائب من عناصر محددة في بنيتها البلورية. 
و من أهم إستخداماتها الإلكترونيات التالية :
\begin{itemize}
	\item 
	الثنائيات الكاشفة للضوء ($  ‫‪Photodiodes‬‬  $) : هو شبه موصل يستعمل في تحويل الضوء إلى كهرباء ، يعمل في مجال الإتصالات ومعالجة الإشارات . \cite{a15}
	\item
	الثنائيات الباعثة للضوء و ثنائيات الليزر ‫‪$ (Light‬‬ ‫‪Emitting‬‬ ‫‪Diodes‬‬ and‬ ‫‪laser‬‬ ‫‪diodes‬‬ ) $ : هو مصدر ضوئي مصنوع من مواد أشباه الموصلات يثير التيار الكهربائي ذراتها ففتشغل بعض إلكتروناتها مستوى طاقة عالي في الذرة. ففي الثنائي الضوئي تقفز إلكترونات الذرّة من مستوى طاقة عالي إلى مستوي طاقة منخفضة متأثرة بالتيار الكهربائي ، فيصدر الإلكترون فارق الطاقة بين الحالتين على هيئة فوتون، أي شعاع ضوء ذو تردد محدد وبالتالي له طول موجة خاص ولون محدد. \cite{a16}
	\item
	الخلايا الكهروضوئية ($  solar cells  $ ) : تعتبر الخلايا الشمسية من أفضل الوسائل لتوليد الطاقة الكهربائية باستخدام أشباه الموصلات لتحويل الطاقة الشمسية إلى تدفق إلكترونات ( كهرباء ). \cite{a17}
\end{itemize}

\begin{figure}[h!]
	\centering
	\includegraphics[width=0.8\linewidth, height=0.25\textheight]{Fig/Fig_I/IMG_20220419_153044}
	\caption{الكترونيات تطبيقية لاشباه الموصلات}
	\label{fig:img20220419153044}
\end{figure}
\FloatBarrier
حيث نعتمد هذه الاستخدامات الدقيقة كبنى تحتية أساسية في وسائل الإعلام الحديثة كشاشات الهواتف و التلفزة ، الكاميرات و إلكترونيات التخزين و الشرائح الإلكترونية ... إلخ .
\begin{figure}[h!]
	\centering
	\includegraphics[width=0.8\linewidth, height=0.4\textheight]{Fig/Fig_I/image02}
	\caption{}
	\label{fig:image02}
\end{figure}
\FloatBarrier

\section { أكبر الشركات العالمية في صناعة الرقائق الالكترونية }

"يعتبر الكثير من المعلقين الأميركيين أن الحرب الباردة الالكترونية قد بدأت بالفعل بين واشنطن وبكين، المتمثلة في تصنيع أشباه الموصلات، وهي رقائق سيليكونية دقيقة تستخدم في الدوائر الإلكترونية لمختلف الأجهزة المستخدمة في الحياة اليومية. "  صحافة الجزيرة : محمد المنشاوي 06/04/2021 \\

وتتقسم هذه الرقائق إلى 3 مستويات، بسيطة ومتوسطة ومتقدمة، وتنتج الكثير من الدول المستويين الأولين، في حين يبقى المستوى الثالث المتطور والمعقد من أجهزة أشباه الموصلات تحت سيطرة عدد صغير من الشركات في عدة دول نذكر منها :

\begin{itemize}
	\item 
	شركة تايوان  $ TSMC $ .
	\item 
	شركة $ INTEL $ .
	\item 
	شركة $ Samsung $ .
	\item 
	شركة $ Huawei $ .
\end{itemize}

\begin{figure}[h!]
	\centering
	\includegraphics[width=0.8\linewidth, height=0.3\textheight]{Fig/Fig_I/screenshot001}
	\caption{مواقع أكبر الشركات العالمية في صناعة الرقائق الالكترونية في العالم . }
	\label{fig:screenshot001}
\end{figure}
\FloatBarrier

\subsection{ النظام الإيكولوجي لأشباه الموصلات }
أشباه الموصلات (الرقائق التي تعالج المعلومات الرقمية) تدخل تقريبا في كل  التحولات الرقمية التي تشهدها حياتنا من : أجهزة الكمبيوتر، والسيارات، والأجهزة المنزلية، والمعدات الطبية،..إلخ .
بالنظر إلى الرسم التوضيحي أدناه  فإن الصناعة تبدو بسيطة جدا، حيث تقوم الشركات في النظام الإيكولوجي لأشباه الموصلات بصنع الرقائق (المثلث على اليسار) وبيعها للشركات والوكالات الحكومية (على اليمين)، ثم تقوم تلك الشركات والوكالات الحكومية بتصميم الرقائق في أنظمة وأجهزة (مثل أجهزة آي فون، وأجهزة الكمبيوتر، والطائرات، والحوسبة السحابية، وغيرها)، ثم تبيعها للمستهلكين والشركات والحكومات، وتبلغ إيرادات المنتجات التي تحتوي على رقائق عشرات التريليونات من الدولارات.

\begin{figure}[h!]
	\centering
	\includegraphics[width=0.8\linewidth, height=0.25\textheight]{Fig/Fig_I/1643121475323(1)}
	\caption{النظام الإيكولوجي لأشباه الموصلات }
	\label{fig:16431214753231}
\end{figure}


في حين هي عملية معقدة تحتاج لمستوى عالي من العلم و التركيز و المادة . فإذا استطعت النظر داخل المثلث البسيط الذي يمثل صناعة أشباه الموصلات، بدلاً من شركة واحدة تصنع الرقائق، فإنك ستجد صناعة تضم مئات الشركات التي تعتمد كلها على بعضها البعض، ويمكننا أن نبدأ بوصف جزء واحد من النظام الإيكولوجي هذه المرة و المتمثل في أكبر الشركات العالمية في صناعة الرقائق الالكترونية  .
\subsection{ شركة تي آس آم سي $ TSMC $  }
\paragraph{تعريفها}
تأسست شركة $ TSMC $ في مدينة $ Hsincho $ في جمهورية الصين الديموقراطية ( تايوان) عام 1987، حيث كانت أول شركة تتأسس في مجال تصنيع أنصاف النواقل. وعبر السنوات تمكنت الشركة من إثبات نفسها كرائدة في المجال  واستفادت بشكل هائل من انفجار شعبية الهواتف الذكية قبل عقد من الزمن، وباتت مهيمنة في المجال اليوم متضمنة شركات عملاقة مثل $ Apple $ و$ AMD $ و $ Nvidia  $ $ Qualcomm $ و  وسواها . \cite{a7}
\paragraph{خدماتها} 
كان الظل المنعكس لهذه الشركة على تطور التكنولوجيا عظيم جدا و ذلك راجع لخدماتها المتعددة نذكر أهمها :
\begin{enumerate}
	\item 
	سيبر شاتل $ CyberShuttle $ :
	تقدم خدمات مرنة ومريحة عبر الإنترنت حيث تعمل من خلال هذه الخدمة على إدارة مخاطر التصميم و 	خفض تكلفة النماذج الأولية بنسبة تصل إلى 90٪ مع 	دعم جميع التقنيات الحديثة .
	
	\item 
	خدمة التغطية الخفيفة  (القناع)   $ Mask Services $ :
	يتم من خلالها تصنيع أقنعة ذات ضمانات و جودة و إنجاز قاعدة بيانات أكثر اكتمالا من $ OPC $ و السيليكون الأمثل مع تقنيات معالجة رقاقة $ TSMC $ . 
	
	كما تقدم خدمات النماذج الأولية $ CyberShuttle $ الخاصة أو المشتركة حيث تشترك العديد من التصاميم في مجموعة قناع واحد لخفض $ NRE $ , ايضا التحويل التنافسي الذي يعزز التسويق السريع .
	
	\item
	التعاون اللوجستي :  تشمل خيارات التعاون الثلاثة لخدمات $ eFoundry $ اي مهام التصميم والهندسة واللوجستيات حيث يوفر الوصول إلى البيانات التي يتم تحديثها يوميا حول حالة الكثير من الرقائق في التصنيع والتجميع والاختبار والنظام والشحن. \cite{a7}
	
\end{enumerate}


\subsection{ شركة سامسونغ $ SAMSUNG $  }

\paragraph{تعريفها}
تأسست يوم 1 مارس 1938 كشركة تجارية على يد إي بيونغ تشول في دايغو، كوريا اليابانية , تعمل على  في مجال الهواتف المحمولة والإلكترونيات الاستهلاكية وأشباه الموصلات. تضم العديد من الشركات التابعة، معظمها متحد تحت علامة سامسونغ التجارية، وهي أكبر شركة تجارية كورية جنوبية (تشايبول). على سبيل مثال هذه التكتلات ، حلول الأجهزة (DS)، وهي شركة أشباه الموصلات فيها، بالاضافة إلى الذاكرة والمسابك والنظام LSI، ولكل منها رئيسها الخاص. \cite{a8}

\paragraph{خدماتها} أقامت شركة سامسونغ مبدأ خدماتها على ملئ نقائص المجال الالكتروني موازاتا مع الاستحقاق الاجتماعي فكانت خدماتها ضمن المجالات اﻵتية :
\begin{enumerate}
	\item
	إلكترونيات سامسونج .
	\item
	سامسونج للتأمين .
	\item
	سامسونج للصناعات الثقيلة .
	\item
	سامسونج للبناء والمقالاوت .
	\item
	سامسونج لتكنولوجيا المعلومات . \cite{a8}
\end{enumerate}

\subsection{ شركة إنتل  Intel }

\paragraph{تعريفها}
إنتل (بالإنجليزية: Intel)‏ تأسست الشركة في عام 1968 كشركة للإلكترونيات المتكاملة ومقرها في (سانتا كلارا، كاليفورنيا، أمريكا)، تأسست على يد روبرت نويس وغوردون مور صاحب الملكية الفكرية لقانون مور ، و هي من أكبر شركات التكنولوجيا متعددة الجنسيات في الولايات المتحدة الأمريكية وهي متخصصة برقائق ومعالجات الكمبيوتر، كما تعتبر من أكبر أسواق صانعي رقائق أشباه الموصلات في العالم . \cite{a9}

\paragraph{منتجاتها}
تنتج شركة انتل العديد من قطع الحاسوب و الالكترونيات على سبيل المثال :
\begin{enumerate}
	\item
	اللوحة الام [Motherboard] و وحدة المعالجة المركزية[Cpu] .
	\item
	كرت الشاشة [Graphiccard] .
	\item
	القرص الصلب [Harddisk] .
	\item
	 ذاكرة الفلاش [Flashmemory] .
	
\end{enumerate}
\paragraph{لمحة عن قانون مور} 
يقصد بقانون مور ( $ Moore's law $ ) التنبؤ الذي وضعه المهندس الأمريكيّ جوردن مور حول قطعة إلكترونيّة تُدعى الترانزستور، وينصّ القانون على أنّ عدد الترانزستورات ( $  Transistors $ ) لكل رقاقة سيليكون يتضاعف كلّ عام. ويقصد بالترانزستورات قطع إلكترونيّة مصنوعة من أشباه الموصلات، تساعد في توليد الإشارات الكهربائية وتضخيمها، والتحكم بها. وضع المهندس مور هذا التبنؤ عام 1965م، واستمر بالملاحظة المضاعفة التي تحدث للترانزستورات، لكن النتيجة كانت أبطأ من المتوقع، فعدّل الإطار الزمني لتوقعه فأصبح عامين، وعلى مدى 50 عامًا من الملاحظة تضاعف عدد الترانزستورات كلّ 18 شهرًا تقريبًا.
إنّ التطوّر الهائل في إنتاج الأجهزة الحاسوبيّة ذات التعقيد والقدرة الحاسوبيّة الكبيرة مع انخفاض التكلفة الماديّة في تصنيعها وبيعها للمستهلك هو أهم التأثيرات الاقتصاديّة لقانون مور.
كما كان له أثر صناعي على أشباه المُوصلات فهو واضح تماماً، حيث ساهم في إنشاء خارطة تنبئيّة للطريق التي تمضي به الصناعة تمتد لمدّة خمسة عقود ابتدأت من 1971م إلى 2020م. 

\begin{figure}[h]
	\centering
	\includegraphics[width=0.5\linewidth]{Fig/Fig_I/figure2}
	\caption{ مخطَّط قانون مور }
	\label{fig:figure2}
\end{figure}
\FloatBarrier

\subsection{ شركة هواوي Huawei }

\paragraph{تعريفها}
تأسست الشركة الصينية هواوي في عام 1987 من قبل رن تشنغ فاي. وفي البداية، كان تركيز هواوي منصبا على تصنيع لوحات ومقاسم الهاتف، ثم وسعت أعمالها لاحقا لتشمل بناء شبكات الاتصالات السلكية واللاسلكية، وتوفير الخدمات التشغيلية والاستشارية   والمعدات للمؤسسات داخل وخارج الصين، وتصنيع أجهزة الاتصالات للسوق الاستهلاكية. \cite{a10}

\paragraph{خدماتها}
تم تنظيم هواوي حول قطاعات أعمال أساسية متعددة نذكرها :
\begin{enumerate}
	\item
	شبكات الاتصالات السلكية واللاسلكية، وبناء شبكات الاتصالات السلكية واللاسلكية والخدمات.
	\item
	المشاريع التجارية، توفير المعدات والبرامج والخدمات لعملاء المؤسسات، على سبيل المثال الحلول الحكومية - انظر Huawei 4G eLTE.
	\item
	أجهزة تصنيع أجهزة الاتصالات الإلكترونية .
	\item
	الأجهزة الالكترمنزلية . \cite{a14}
\end{enumerate}
