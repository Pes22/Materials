\chapter*{المقدمة} % 

\label{Chapter0} 
%----------------------------------------------------------------------------------------

% Define some commands to keep the formatting separated from the content 
\newcommand{\keyword}[1]{\textbf{#1}}
\newcommand{\tabhead}[1]{\textbf{#1}}
\newcommand{\code}[1]{\texttt{#1}}
%\newcommand{\file}[1]{\texttt{\bfseries#1}}
%\newcommand{\option}[1]{\texttt{\itshape#1}}

%----------------------------------------------------------------------------------------

لفيزياء المواد الصلبة أهمية تكنولوجية جليلة ، حيث طورت خصائص هذه المواد لتستعمل في العديد من المجالات . فعند استخدامنا للمواد الصلبة في الصناعة يجب علينا مراعاة جودة الخصائص الإلكترونية ، البصرية و المغناطيسية ... إلخ.

من بين هذه المواد أشباه النواقل و التي أحدثت ضوضاء علمية مأخرا و اعتبرها بعض المعلقين الفيزيائيين الحرب الباردة في زيها الحديث و ذلك راجع لتطبيقاتها الالكترونية في مجال الذكاء الإصطناعي و الطاقات المتجددة .\\

زيادة على ذلك مزاياها العظمى مثل زيادة كفاءة الترانزستور و النبائط ذات الطرفين كالمقومات والثنائيات الضوئية أو سرعة معالجة البيانات في الشرائح الإلكترونية ، لما تتميز به من خصائص فريدة عند موصيليتها، خصوصا بعد إضافة شوائب من عناصر محددة في بنيتها البلورية مما يحسن قيم فجوات الطاقة وتردد الأنماط البصرية والسماحية و مجموعة النتائج التي تسهل تطبيقاتها الصناعية و المختبرية . \\

من بين المواد الشبه موصلة عناصر الأرض الناذرة التي أثبت وجودها منذ == ...........
\\
في هذه المذكرة قمنا بدراسة الخصائص الإلكترونية ، البصرية و المغناطيسية لمركب نتريد الغاليوم $ GaN $ و أثر تطعيمه بعنصر الاوروبيوم مشكلا مركب $ EuGaN $ . حيث تم الحصول على النتائج في سياق نظرية دالية الكثافة $ DFT $ وذلك باستخدام طريقة الموجة المستوية المتزايدة خطيا والكمون الكامل $ (FP-LAPW) $ والمدمجة في برنامج $ Wien2k $ .

حيث تطرقنا في هذا العمل الى جزئين أساسين :
\begin{itemize}
	\item 
	جزء نظري : شمل عموميات حول أشباه النواقل في الفصل الأول بالإضافة الى برنامج  و دمجه لدالة الكثافة و طريقة الأمواج المستوية المتزايدة خطيا .
\end{itemize}
 جزء تطبيقي : استخدمنا فيه النهج الحسابي للمبدأ الأول القائم على نظرية الكثافة الوظيفية وتوقعنا السلوك الفعال في تحسين الخصائص لأشباه النواقل المطعمة بأحد عناصر الأرض الناذرة ، حيث استخدمنا برنامج Wien2k للتحقق من نتائج النظرية عن طريق الأمواج المستوية المتزايدة خطيا في دراسة الخصائص الإلكترونية و الضوئية لمركب نتريد الغاليوم و مقارنتها بخصائصه الجديدة بعد تطعيمه بعنصر الأوروبيوم للحصول على النتائج الرجعية لعملية التطعيم على خصائص أشباه النواقل ثم مناقشتها و مقارنتها بالدراسات السابقة .


الكلمات المفتاحية:
أشباه النواقل ، دالة الكثافة ، طريقة الأمواج المستوية المتزايدة خطيا ، wien2k .

بسم الله الرحمان الرحيم

رب أوزعني أن أشكر نعمتك التي أنعمت علي وعلى والدي وأن أعمل صالحا ترضاه .

أهدي ثمرة هذا العمل إلى :

الذين وصى الله بهما إحسانا

إلى والدي

إلى الذين تقر بهم الأعين 
إلى إخوتي و أسرهم الصغيرة .
إلى من تحملني وتحملته فكان نتاج ذلك هذا العمل.
إلى من تمنوا رؤيتي في هذا الموقف لكن مشيئة الله سبقت رحمة الله عليهم. 
إلى فلسطين الحبيبة والقدس الحرة عاصمتها الأبدية. 
الى المدرسة الام اللجنة التنسيقية للمساجد الجامعية بشار، إلى خدام الجامعة و مناضليها .
إلى رفيقات الدرب أخواتي وصديقاتي كل باسمها. 
إلى طلاب العلم وهواة المعرفة في كل مكان...
و آخرا الى جميع من ساند في إتمام هذا البحث .


تشكرات

تكمل سعادتي وأنا أضع آخر لمسات هذا البحث المتواضع ، أن أشكر الله عز وجل على توفيقه في إتمامه ، وان أسدي الشكر لمستحقيه فلولاهم ما اكتمل هذا البحث .

أتوجه بجزيل الشكر والامتنان إلى الأستاذ الذي نمى في حب العلم و المعرفة الدكتور عميري بن عامر الذي لم يبخل علينا بجميل معرفته و توجيهاته ونصائحه  التي كانت عونا لنا في كل عويصة.
واشكر كل من ساعدني على إتمام هذه المذكرة وقدم لي الدعم كيفما كان .
كما لا يفوتني العرفان بمجهودات المدرسة العليا للأساتذة ، بشار على تفانيها في تخرج دفعات
كما لا أنسي الأساتذة أعضاء اللجنة على تفضلهم لقبول تقييمهم هذا العمل

المتواضع و تصحيحه.

إلى العائلة العباسية العريقة عائلة تهامي .

إلى زملاء دفعة استاذ تعليم ثانوي تخصص علوم فيزيائية  2021/ 2022

وأسأل الله التوفيق لكل هواة العلم و المعرفة في كل مكان .