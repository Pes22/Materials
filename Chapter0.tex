\chapter*{المقدمة} % 

\label{Chapter0} 
%----------------------------------------------------------------------------------------

% Define some commands to keep the formatting separated from the content 
\newcommand{\keyword}[1]{\textbf{#1}}
\newcommand{\tabhead}[1]{\textbf{#1}}
\newcommand{\code}[1]{\texttt{#1}}
%\newcommand{\file}[1]{\texttt{\bfseries#1}}
%\newcommand{\option}[1]{\texttt{\itshape#1}}

%----------------------------------------------------------------------------------------

لفيزياء المواد الصلبة أهمية تكنولوجية جليلة ، حيث طورت خصائص هذه المواد لتستعمل في العديد من المجالات . فعند استخدامنا للمواد الصلبة في الصناعة يجب علينا مراعاة جودة الخصائص الإلكترونية ، البصرية و المغناطيسية ... إلخ.

من بين هذه المواد أشباه النواقل و التي أحدثت ضوضاء علمية مأخرا و اعتبرها بعض المعلقين الفيزيائيين الحرب الباردة في زيها الحديث و ذلك راجع لتطبيقاتها الالكترونية في مجال الذكاء الإصطناعي و الطاقات المتجددة .\\

زيادة على ذلك مزاياها العظمى مثل زيادة كفاءة الترانزستور و النبائط ذات الطرفين كالمقومات والثنائيات الضوئية أو سرعة معالجة البيانات في الشرائح الإلكترونية ، لما تتميز به من خصائص فريدة عند موصيليتها، خصوصا بعد إضافة شوائب من عناصر محددة في بنيتها البلورية مما يحسن قيم فجوات الطاقة وتردد الأنماط البصرية والسماحية و مجموعة النتائج التي تسهل تطبيقاتها الصناعية و المختبرية . \\

من بين المواد الشبه موصلة عناصر الأرض الناذرة التي أثبت وجودها منذ == ...........
\\
في هذه المذكرة قمنا بدراسة الخصائص الإلكترونية ، البصرية و المغناطيسية لمركب نتريد الغاليوم $ GaN $ و أثر تطعيمه بعنصر الاوروبيوم مشكلا مركب $ EuGaN $ . حيث تم الحصول على النتائج في سياق نظرية دالية الكثافة $ DFT $ وذلك باستخدام طريقة الموجة المستوية المتزايدة خطيا والكمون الكامل $ (FP-LAPW) $ والمدمجة في برنامج $ Wien2k $ .

الملخص :

في هذا العمل تطرقنا الى جزئيين أساسين جزء نظري: شمل عموميات عن أشباه النواقل III-V بالإضافة الى دالية الكثافة والأمواج المستوية المتزايدة خطيا، و جزء تطبيقي استخدمنا فيه النهج الحسابي للمبدأ الأول القائم على نظرية الكثافة الوظيفية المستقطبة وتوقعنا السلوك شبه المعدني لسبائك( n-xMSb (M=Feاباستخدام برنامج Wien2k، من أجل التراكيب المختلفة 0.5=x تتم دراسة الخواص الهيكلية والإلكترونية والمغناطيسية لهذه البلورات المكعبة الثلاثية والحصول على النتائج ومناقشتها .

الكلمات المفتاحية:

أشباه النواقل دالية الكثافة - دالية الأمواج المستوية المتزايدة خطيا - سلوك شبه المعدني لسبائك ((0.5=wien2k - In, M, Sb (M =Fe, x.
