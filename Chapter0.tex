\chapter*{المقدمة} % 

\label{Chapter0} 
%----------------------------------------------------------------------------------------

% Define some commands to keep the formatting separated from the content 
\newcommand{\keyword}[1]{\textbf{#1}}
\newcommand{\tabhead}[1]{\textbf{#1}}
\newcommand{\code}[1]{\texttt{#1}}
%\newcommand{\file}[1]{\texttt{\bfseries#1}}
%\newcommand{\option}[1]{\texttt{\itshape#1}}

%----------------------------------------------------------------------------------------


‫ﺗﻮﺟﺪ ‫اﳌﺎدة‬  ‬في  ‫ﺣﺎﻻت‬ ‫أرﺑﻌﺔ‬ هي : الحالة الصلبة و السائلة و الغازية أو البلازما , و يختلف ذلك من حالة لأخرى حسب البنية البلورية للمادة و قوى الربط بين جزيئاتها و تقبل كل حالة التحول  إلى أخرى تحت تأثير عوامل فيزيائية مختلفة ( P , T , V … ) .
كما تصنف المادة وفقا لخصائصها الكهربائية الى ثلاث , النواقل و العوازل و أشباه النواقل حيث تشكل هذه الاخيرة قاعدة أساسية للبنى التحتية الالكترنية و التى اصبح العالم اليوم يطالب بالاطلاع على  تعريفاتها, مبادئها و تطبيقاته \cite{1,2}.

